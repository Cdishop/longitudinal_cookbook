\documentclass[]{article}
\usepackage{lmodern}
\usepackage{amssymb,amsmath}
\usepackage{ifxetex,ifluatex}
\usepackage{fixltx2e} % provides \textsubscript
\ifnum 0\ifxetex 1\fi\ifluatex 1\fi=0 % if pdftex
  \usepackage[T1]{fontenc}
  \usepackage[utf8]{inputenc}
\else % if luatex or xelatex
  \ifxetex
    \usepackage{mathspec}
  \else
    \usepackage{fontspec}
  \fi
  \defaultfontfeatures{Ligatures=TeX,Scale=MatchLowercase}
\fi
% use upquote if available, for straight quotes in verbatim environments
\IfFileExists{upquote.sty}{\usepackage{upquote}}{}
% use microtype if available
\IfFileExists{microtype.sty}{%
\usepackage{microtype}
\UseMicrotypeSet[protrusion]{basicmath} % disable protrusion for tt fonts
}{}
\usepackage[margin=1in]{geometry}
\usepackage{hyperref}
\hypersetup{unicode=true,
            pdfborder={0 0 0},
            breaklinks=true}
\urlstyle{same}  % don't use monospace font for urls
\usepackage{longtable,booktabs}
\usepackage{graphicx,grffile}
\makeatletter
\def\maxwidth{\ifdim\Gin@nat@width>\linewidth\linewidth\else\Gin@nat@width\fi}
\def\maxheight{\ifdim\Gin@nat@height>\textheight\textheight\else\Gin@nat@height\fi}
\makeatother
% Scale images if necessary, so that they will not overflow the page
% margins by default, and it is still possible to overwrite the defaults
% using explicit options in \includegraphics[width, height, ...]{}
\setkeys{Gin}{width=\maxwidth,height=\maxheight,keepaspectratio}
\IfFileExists{parskip.sty}{%
\usepackage{parskip}
}{% else
\setlength{\parindent}{0pt}
\setlength{\parskip}{6pt plus 2pt minus 1pt}
}
\setlength{\emergencystretch}{3em}  % prevent overfull lines
\providecommand{\tightlist}{%
  \setlength{\itemsep}{0pt}\setlength{\parskip}{0pt}}
\setcounter{secnumdepth}{0}
% Redefines (sub)paragraphs to behave more like sections
\ifx\paragraph\undefined\else
\let\oldparagraph\paragraph
\renewcommand{\paragraph}[1]{\oldparagraph{#1}\mbox{}}
\fi
\ifx\subparagraph\undefined\else
\let\oldsubparagraph\subparagraph
\renewcommand{\subparagraph}[1]{\oldsubparagraph{#1}\mbox{}}
\fi

%%% Use protect on footnotes to avoid problems with footnotes in titles
\let\rmarkdownfootnote\footnote%
\def\footnote{\protect\rmarkdownfootnote}

%%% Change title format to be more compact
\usepackage{titling}

% Create subtitle command for use in maketitle
\newcommand{\subtitle}[1]{
  \posttitle{
    \begin{center}\large#1\end{center}
    }
}

\setlength{\droptitle}{-2em}

  \title{}
    \pretitle{\vspace{\droptitle}}
  \posttitle{}
    \author{}
    \preauthor{}\postauthor{}
    \date{}
    \predate{}\postdate{}
  

\begin{document}

Different literatures talk about the idea of level in different ways. In
mathematical biology a variable is referred to as a state, the span of
values that it can take is known as the state space, and its current
value is called the condition of the state. Level is therefore
synonymous with the condition of a state. Monge refers to it as
magnitude.

In the statistical literature, level is called the intercept. Level,
with respect to a certain point or window in time, is called the
intercept in the statistical literature, and we can make an inference
about the intercept at any observation point. Sometimes we are
interested in the level at the beginning of the study, other times we
are interested in the level at the last observation. We can of course
also be interested in the average level across time.

\hypertarget{level}{%
\section{Level}\label{level}}

Is employee emotional exhaustion, on average, high across the study? Is
trainee skill low at the beginning of a training session? What value are
newcomer perceptions of unit climate at the end of a two-week
socialization process? These are questions about level, or the specific
value of a variable.

We can think about level at a specific moment or averaged across a
window of time. That is, if we put a variable on the \(y\) axis and plot
its values against time on the \(x\) axis, we can explore the value that
it takes at time \(t\), or the value that it takes on average across any
span of \(t\). Figure \ref{level} demonstrates this idea graphically. A
variable is plotted across time, and the color labels indicate levels --
two at specific \(t\)'s and the third averaged across time -- that we
may be interested in.

\begin{center}

------------

Insert Figure \ref{level} about here

------------

\end{center}

\noindent Our first level inference, therefore, concerns the value of a
variable at a specific time or averaged across a window of time.

\hypertarget{inference-1-what-is-the-level-of-x-at-time-t-or-across-a-span-of-t}{%
\paragraph{\texorpdfstring{Inference 1: What is the level of \(x\) at
time \(t\), or across a span of
\(t\)?}{Inference 1: What is the level of x at time t, or across a span of t?}}\label{inference-1-what-is-the-level-of-x-at-time-t-or-across-a-span-of-t}}

When we retain one variable but add multiple units -- people or
organizations, for example -- then we can look at the variability in
level. Does everyone have high affect across time? Is there variability
in the level of skill among trainees at the beginning of a training
session? We demonstrate this idea in figure \ref{level_var}, where each
unit (person) has a similar trajectory but different levels at the last
time point.

\begin{center}

------------

Insert Figure \ref{level_var} about here

------------

\end{center}

\noindent The second level inference, therefore, is about the
variability of level across units.

\hypertarget{inference-2-there-is-variability-in-the-level-of-x-at-time-t-or-across-a-span-of-t.}{%
\paragraph{\texorpdfstring{Inference 2: There is variability in the
level of \(x\) at time \(t\), or across a span of
\(t\).}{Inference 2: There is variability in the level of x at time t, or across a span of t.}}\label{inference-2-there-is-variability-in-the-level-of-x-at-time-t-or-across-a-span-of-t.}}

Inferences one and two concern a single variable, but they can of course
be iterated across any or all observed variables in a study. For
example, we might discover that \(x\) and \(y\) have high average levels
across time, but that \(y\) has greater variability, suggesting
individual differences in the sample. Or we might learn that \(x\) has a
low initial level whereas \(y\)'s initial level is high.

Correlating these levels is the next inference. When \(x\) on average is
low at the initial time point and \(y\) on average is high at the
initial time point, we can correlate them to discover if units (people)
with low initial levels of \(x\) have high initial levels on \(y\) and
people with high initial levels on \(x\) have low initial levels on
\(y\).

Figure \ref{level_correlate} demonstrates this graphically. all about
the graph.

\hypertarget{inference-3-there-is-a-correlation-between-the-level-of-x-and-y-at-t.}{%
\paragraph{\texorpdfstring{Inference 3: There is a correlation between
the level of \(x\) and \(y\) at
\(t\).}{Inference 3: There is a correlation between the level of x and y at t.}}\label{inference-3-there-is-a-correlation-between-the-level-of-x-and-y-at-t.}}

The final inference is horizontal. Rather than regressing a point
estimate of the level on another level, we regress the values on a
variable across time on another variable's values across time.

Questions. Is the level of affect related to the level of helping
behaviors across time? That is, when affect is high, are helping
behaviors also high or are they low? When team cohesion is low, is team
performance low as well, or is it high?

Figure \ref{level_relation}

\hypertarget{inference-4-there-is-a-relationship-between-x-and-y-across-time.}{%
\paragraph{\texorpdfstring{Inference 4: There is a relationship between
\(x\) and \(y\) across
time.}{Inference 4: There is a relationship between x and y across time.}}\label{inference-4-there-is-a-relationship-between-x-and-y-across-time.}}

\hypertarget{level-inference-table}{%
\subsection{Level Inference Table}\label{level-inference-table}}

\% latex table generated in R 3.5.1 by xtable 1.8-2 package \% Sun Sep
30 18:09:56 2018

\begin{table}[ht]
\centering
\begin{tabular}{rll}
  \hline
 & Inference & Examples \\ 
  \hline
1 & What is the level of \$x\$ at \$t\$, or across a span of \$t\$? & Burnout is high at the last time point.$\backslash$nPerformance is low, on average, across time. \\ 
  2 & There is variability in the level of \$x\$ at \$t\$, or across a span of \$t\$. & Average affect across time differs across people (units).$\backslash$nThere is variability in the initial level of turnover across organizations. \\ 
  3 & There is a correlation between the level of \$x\$ and the level of \$y\$ at \$t\$. & People with greater initial health status also have greater initial happiness.$\backslash$nPeople with high performance on average across time have lower anxiety on average across time. \\ 
  4 & There is a relationship between \$x\$ and \$y\$ across time. & Affect relates to performance across time.$\backslash$nHelping behaviors predict depletion across time. \\ 
   \hline
\end{tabular}
\end{table}

\begin{longtable}[]{@{}ll@{}}
\toprule
Inference & Examples\tabularnewline
\midrule
\endhead
What is the level of \(x\) at \(t\), or across a span of \(t\)? &
Burnout is high at the last time point.\nPerformance is low, on average,
across time.\tabularnewline
There is variability in the level of \(x\) at \(t\), or across a span of
\(t\). & Average affect across time differs across people
(units).\nThere is variability in the initial level of turnover across
organizations.\tabularnewline
There is a correlation between the level of \(x\) and the level of \(y\)
at \(t\). & People with greater initial health status also have greater
initial happiness.\nPeople with high performance on average across time
have lower anxiety on average across time.\tabularnewline
There is a relationship between \(x\) and \(y\) across time. & Affect
relates to performance across time.\nHelping behaviors predict depletion
across time.\tabularnewline
\bottomrule
\end{longtable}

\begin{table}[tbp]
\begin{center}
\begin{threeparttable}
\caption{\label{tab:unnamed-chunk-1}}
\begin{tabular}{ll}
\toprule
Inference & \multicolumn{1}{c}{Examples}\\
\midrule
What is the level of \$x\$ at \$t\$, or across a span of \$t\$? & Burnout is high at the last time point.\textbackslash{}nPerformance is low, on average, across time.\\
There is variability in the level of \$x\$ at \$t\$, or across a span of \$t\$. & Average affect across time differs across people (units).\textbackslash{}nThere is variability in the initial level of turnover across organizations.\\
There is a correlation between the level of \$x\$ and the level of \$y\$ at \$t\$. & People with greater initial health status also have greater initial happiness.\textbackslash{}nPeople with high performance on average across time have lower anxiety on average across time.\\
There is a relationship between \$x\$ and \$y\$ across time. & Affect relates to performance across time.\textbackslash{}nHelping behaviors predict depletion across time.\\
\bottomrule
\end{tabular}
\end{threeparttable}
\end{center}
\end{table}

What is the level of \(x\) at \(t\), or across a span of \(t\)?

Burnout is high at the last time point. Performance is low, on average,
across time.

There is variability in the level of \(x\) at \(t\), or across a span of
\(t\).

Average affect across time differs across people (units). There is
variability in the initial level of turnover across organizations.

There is a correlation between the level of \(x\) and level of \(y\) at
\(t\).

People with greater initial health status also have greater initial
happiness.

The level of \(x\) at time \(t\) is\ldots{} Picture 1.

There is variability in the level of \(x\) at time \(t\). Picture 2.

Correlate levels. People with high starting point on \(x\) have a high
starting point on \(y\).

There is a relationship between the levels of multiple variables. \(x\)
relates to \(y\) across time. Stable vs fluctuating \(x\),
time-invariant vs time-varying covariates.

\hypertarget{models}{%
\subsection{Models}\label{models}}

Intercept only models: these can be done in HLM or SEM. Time-varying or
invariant covariates analysis, these can be done in HLM or SEM. Point to
references.

\begin{verbatim}
## `geom_smooth()` using method = 'loess' and formula 'y ~ x'
\end{verbatim}

\begin{verbatim}
## Warning in simpleLoess(y, x, w, span, degree = degree, parametric =
## parametric, : span too small. fewer data values than degrees of freedom.
\end{verbatim}

\begin{verbatim}
## Warning in simpleLoess(y, x, w, span, degree = degree, parametric =
## parametric, : pseudoinverse used at -0.1
\end{verbatim}

\begin{verbatim}
## Warning in simpleLoess(y, x, w, span, degree = degree, parametric =
## parametric, : neighborhood radius 10.1
\end{verbatim}

\begin{verbatim}
## Warning in simpleLoess(y, x, w, span, degree = degree, parametric =
## parametric, : reciprocal condition number 0
\end{verbatim}

\begin{verbatim}
## Warning in simpleLoess(y, x, w, span, degree = degree, parametric =
## parametric, : There are other near singularities as well. 102.01
\end{verbatim}

\begin{verbatim}
## Warning: Removed 2 rows containing missing values (geom_point).
\end{verbatim}

\begin{figure}
\centering
\includegraphics{lc_level_files/figure-latex/unnamed-chunk-2-1.pdf}
\caption{Level examples\label{level}}
\end{figure}

One variable, multiple units

\begin{verbatim}
## `geom_smooth()` using method = 'loess' and formula 'y ~ x'
\end{verbatim}

\begin{verbatim}
## Warning in simpleLoess(y, x, w, span, degree = degree, parametric =
## parametric, : span too small. fewer data values than degrees of freedom.
\end{verbatim}

\begin{verbatim}
## Warning in simpleLoess(y, x, w, span, degree = degree, parametric =
## parametric, : pseudoinverse used at -0.1
\end{verbatim}

\begin{verbatim}
## Warning in simpleLoess(y, x, w, span, degree = degree, parametric =
## parametric, : neighborhood radius 10.1
\end{verbatim}

\begin{verbatim}
## Warning in simpleLoess(y, x, w, span, degree = degree, parametric =
## parametric, : reciprocal condition number 0
\end{verbatim}

\begin{verbatim}
## Warning in simpleLoess(y, x, w, span, degree = degree, parametric =
## parametric, : There are other near singularities as well. 102.01
\end{verbatim}

\begin{verbatim}
## Warning in simpleLoess(y, x, w, span, degree = degree, parametric =
## parametric, : span too small. fewer data values than degrees of freedom.
\end{verbatim}

\begin{verbatim}
## Warning in simpleLoess(y, x, w, span, degree = degree, parametric =
## parametric, : pseudoinverse used at -0.1
\end{verbatim}

\begin{verbatim}
## Warning in simpleLoess(y, x, w, span, degree = degree, parametric =
## parametric, : neighborhood radius 10.1
\end{verbatim}

\begin{verbatim}
## Warning in simpleLoess(y, x, w, span, degree = degree, parametric =
## parametric, : reciprocal condition number 0
\end{verbatim}

\begin{verbatim}
## Warning in simpleLoess(y, x, w, span, degree = degree, parametric =
## parametric, : There are other near singularities as well. 102.01
\end{verbatim}

\begin{verbatim}
## Warning in simpleLoess(y, x, w, span, degree = degree, parametric =
## parametric, : span too small. fewer data values than degrees of freedom.
\end{verbatim}

\begin{verbatim}
## Warning in simpleLoess(y, x, w, span, degree = degree, parametric =
## parametric, : pseudoinverse used at -0.1
\end{verbatim}

\begin{verbatim}
## Warning in simpleLoess(y, x, w, span, degree = degree, parametric =
## parametric, : neighborhood radius 10.1
\end{verbatim}

\begin{verbatim}
## Warning in simpleLoess(y, x, w, span, degree = degree, parametric =
## parametric, : reciprocal condition number 0
\end{verbatim}

\begin{verbatim}
## Warning in simpleLoess(y, x, w, span, degree = degree, parametric =
## parametric, : There are other near singularities as well. 102.01
\end{verbatim}

\begin{verbatim}
## Warning: Removed 12 rows containing missing values (geom_point).
\end{verbatim}

\begin{figure}
\centering
\includegraphics{lc_level_files/figure-latex/unnamed-chunk-3-1.pdf}
\caption{Trajectories with similar starting and average levels but
different ending levels\label{level_var}}
\end{figure}

Muliple Variables

When affect is low performance is low. When affect is high performance
is high.

\begin{verbatim}
## -- Attaching packages -------------------------------------------------------------------------- tidyverse 1.2.1 --
\end{verbatim}

\begin{verbatim}
## √ tibble  1.4.2     √ purrr   0.2.5
## √ tidyr   0.8.1     √ dplyr   0.7.6
## √ readr   1.1.1     √ stringr 1.3.1
## √ tibble  1.4.2     √ forcats 0.3.0
\end{verbatim}

\begin{verbatim}
## -- Conflicts ----------------------------------------------------------------------------- tidyverse_conflicts() --
## x dplyr::filter() masks stats::filter()
## x dplyr::lag()    masks stats::lag()
\end{verbatim}

\begin{verbatim}
## 
## Attaching package: 'gridExtra'
\end{verbatim}

\begin{verbatim}
## The following object is masked from 'package:dplyr':
## 
##     combine
\end{verbatim}

\begin{figure}
\centering
\includegraphics{lc_level_files/figure-latex/unnamed-chunk-4-1.pdf}
\caption{Relating affect to performance levels\label{level_relation}}
\end{figure}


\end{document}
