\documentclass[english,,man]{apa6}
\usepackage{lmodern}
\usepackage{amssymb,amsmath}
\usepackage{ifxetex,ifluatex}
\usepackage{fixltx2e} % provides \textsubscript
\ifnum 0\ifxetex 1\fi\ifluatex 1\fi=0 % if pdftex
  \usepackage[T1]{fontenc}
  \usepackage[utf8]{inputenc}
\else % if luatex or xelatex
  \ifxetex
    \usepackage{mathspec}
  \else
    \usepackage{fontspec}
  \fi
  \defaultfontfeatures{Ligatures=TeX,Scale=MatchLowercase}
\fi
% use upquote if available, for straight quotes in verbatim environments
\IfFileExists{upquote.sty}{\usepackage{upquote}}{}
% use microtype if available
\IfFileExists{microtype.sty}{%
\usepackage{microtype}
\UseMicrotypeSet[protrusion]{basicmath} % disable protrusion for tt fonts
}{}
\usepackage{hyperref}
\hypersetup{unicode=true,
            pdftitle={Inferences With Longitudinal Data},
            pdfauthor={Christopher R. Dishop, Michael T. Braun, Goran Kuljanin, \& Richard P. DeShon},
            pdfkeywords={longitudinal inferences, growth trends, dynamics, relationships over
time, processes},
            pdfborder={0 0 0},
            breaklinks=true}
\urlstyle{same}  % don't use monospace font for urls
\ifnum 0\ifxetex 1\fi\ifluatex 1\fi=0 % if pdftex
  \usepackage[shorthands=off,main=english]{babel}
\else
  \usepackage{polyglossia}
  \setmainlanguage[]{english}
\fi
\usepackage{graphicx,grffile}
\makeatletter
\def\maxwidth{\ifdim\Gin@nat@width>\linewidth\linewidth\else\Gin@nat@width\fi}
\def\maxheight{\ifdim\Gin@nat@height>\textheight\textheight\else\Gin@nat@height\fi}
\makeatother
% Scale images if necessary, so that they will not overflow the page
% margins by default, and it is still possible to overwrite the defaults
% using explicit options in \includegraphics[width, height, ...]{}
\setkeys{Gin}{width=\maxwidth,height=\maxheight,keepaspectratio}
\IfFileExists{parskip.sty}{%
\usepackage{parskip}
}{% else
\setlength{\parindent}{0pt}
\setlength{\parskip}{6pt plus 2pt minus 1pt}
}
\setlength{\emergencystretch}{3em}  % prevent overfull lines
\providecommand{\tightlist}{%
  \setlength{\itemsep}{0pt}\setlength{\parskip}{0pt}}
\setcounter{secnumdepth}{0}
% Redefines (sub)paragraphs to behave more like sections
\ifx\paragraph\undefined\else
\let\oldparagraph\paragraph
\renewcommand{\paragraph}[1]{\oldparagraph{#1}\mbox{}}
\fi
\ifx\subparagraph\undefined\else
\let\oldsubparagraph\subparagraph
\renewcommand{\subparagraph}[1]{\oldsubparagraph{#1}\mbox{}}
\fi

%%% Use protect on footnotes to avoid problems with footnotes in titles
\let\rmarkdownfootnote\footnote%
\def\footnote{\protect\rmarkdownfootnote}


  \title{Inferences With Longitudinal Data}
    \author{Christopher R. Dishop\textsuperscript{1}, Michael T.
Braun\textsuperscript{2}, Goran Kuljanin\textsuperscript{3}, \& Richard
P. DeShon\textsuperscript{1}}
    \date{}
  
\shorttitle{LONGITUDINAL INFERENCES}
\affiliation{
\vspace{0.5cm}
\textsuperscript{1} Michigan State University\\\textsuperscript{2} University of South Florida\\\textsuperscript{3} DePaul University}
\keywords{longitudinal inferences, growth trends, dynamics, relationships over time, processes\newline\indent Word count: 151}
\usepackage{csquotes}
\usepackage{upgreek}
\captionsetup{font=singlespacing,justification=justified}

\usepackage{longtable}
\usepackage{lscape}
\usepackage{multirow}
\usepackage{tabularx}
\usepackage[flushleft]{threeparttable}
\usepackage{threeparttablex}

\newenvironment{lltable}{\begin{landscape}\begin{center}\begin{ThreePartTable}}{\end{ThreePartTable}\end{center}\end{landscape}}

\makeatletter
\newcommand\LastLTentrywidth{1em}
\newlength\longtablewidth
\setlength{\longtablewidth}{1in}
\newcommand{\getlongtablewidth}{\begingroup \ifcsname LT@\roman{LT@tables}\endcsname \global\longtablewidth=0pt \renewcommand{\LT@entry}[2]{\global\advance\longtablewidth by ##2\relax\gdef\LastLTentrywidth{##2}}\@nameuse{LT@\roman{LT@tables}} \fi \endgroup}


\DeclareDelayedFloatFlavor{ThreePartTable}{table}
\DeclareDelayedFloatFlavor{lltable}{table}
\DeclareDelayedFloatFlavor*{longtable}{table}
\makeatletter
\renewcommand{\efloat@iwrite}[1]{\immediate\expandafter\protected@write\csname efloat@post#1\endcsname{}}
\makeatother
\usepackage{lineno}

\linenumbers

\authornote{\ldots{}.

Correspondence concerning this article should be addressed to
Christopher R. Dishop, 316 Physics Road, Psychology Building, Room 348,
East Lansing, MI, 48823. E-mail:
\href{mailto:dishopch@msu.edu}{\nolinkurl{dishopch@msu.edu}}}

\abstract{
Organizational scientists recognize that psychological phenomena and
processes unfold over time. To better understand psychological phenomena
over time, organizational researchers increasingly work with
longitudinal data and explore inferences within those data structures.
Longitudinal inferences may focus on any number of fundamental patterns,
including construct trajectories, relationships between constructs, or
dynamics. Although the diversity of longitudinal inferences provides a
wide foundation for garnering knowledge in any given area, it also makes
it difficult for researchers to know the set of inferences they may
explore with longitudinal data, which statistical models to use given
their question, and how to locate their specific study within the
broader set of longitudinal inferences. In this paper, we develop a
framework to describe the variety of between-unit research questions and
inferences researchers may explore with longitudinal data and link those
inferences to statistical models so researchers know where to turn to
given their particular interests.


}

\usepackage{amsthm}
\newtheorem{theorem}{Theorem}[section]
\newtheorem{lemma}{Lemma}[section]
\theoremstyle{definition}
\newtheorem{definition}{Definition}[section]
\newtheorem{corollary}{Corollary}[section]
\newtheorem{proposition}{Proposition}[section]
\theoremstyle{definition}
\newtheorem{example}{Example}[section]
\theoremstyle{definition}
\newtheorem{exercise}{Exercise}[section]
\theoremstyle{remark}
\newtheorem*{remark}{Remark}
\newtheorem*{solution}{Solution}
\begin{document}
\maketitle

Organizational scientists recognize that psychological phenomena and
processes unfold over time (Beal, 2015; Pitariu \& Ployhart, 2010).
Individuals in the workplace, over time, strive to accomplish work
goals, team members collaborate so the whole eventually becomes greater
than the sum of its parts, and managers repeatedly promote values to
build vibrant, innovative work cultures. To better understand
psychological phenomena, such as motivation, teamwork, and
organizaitonal culture, researchers must attent not to static snapshots
of behavior (Ilgen \& Hulin, 2000; Kozlowski, Chao, Grand, Braun, \&
Kuljanin, 2013, 2016) but to variables and relationships as they move
through time. Obtaining longitudinal data allows researchers to capture
the unfolding set of events, interactions, behaviors, cognitions, or
affective reactions across a variety of psychological phenomena.

Researchers have the opportunity to explore many inferences when they
analyze longitudinal data. For example, researchers may examine the
shape of trajectories on psychological constructs (e.g., Did job
satisfaction generally increase or decrease during the past six
months?), how two or more constructs relate to each other (e.g., Did
team communication and cohesion positively correlate over time?), or
whether changes in one variable relate to changes in another (e.g., Did
changes in goal-setting relate to changes in employee performance?
Dunford, Shipp, Boss, Angermeier, \& Boss, 2012; Hardy, Day, \& Steele,
2018; Jones et al., 2016; Judge, Simon, Hurst, \& Kelley, 2014; Lanaj,
Johnson, \& Wang, 2016; Rosen, Koopman, Gabriel, \& Johnson, 2016; Scott
\& Barnes, 2011). Given the variety of available inferences with
longitudinal data, an organizing framework would elucidate their subtle
differences, enhance theoretical insight, guide data collection, and
facilitate sound analytical work.

We developed a framework to capture these inferences, a way to organize
the fundamental between-unit patterns researchers explore with
longitudinal data despite focusing on different content areas or using
different statistical models. Researchers often focus on one famililiar
inference despite having the data to explore many more fundamental
patterns. We bring attention to the span of questions available so that
researchers can fully appreciate and take advantage of their data.
Moreover, there are many complex statistical models lingering in our
literature and it is not always clear for which questions they are
appropriate. We provide readers with potential models for each inference
so that they can be sure that the model they evoke is appropriate for
the research question that they are interested in. In summary, this
paper exposes researchers to the span of between-unit inferences they
may investigate when they collect longitudinal data, links those
inferences to statistical models, and explains differences between
various longitudinal inferences.

\hypertarget{longitudinal-research-in-applied-psychology}{%
\section{Longitudinal Research in Applied
Psychology}\label{longitudinal-research-in-applied-psychology}}

This paper is devoted to inferences with repeated measures, so we begin
with a few labels and definitions. Authors typically identify a
\enquote{longitudinal} study by contrasting either (a) research designs
or (b) data structures. Longitudinal \emph{research} is different from
cross-sectional research because longitudinal designs entail three or
more repeated observations (Ployhart \& Vandenberg, 2010). We therefore
emphasize differences on the number of observations when we distinguish
longitudinal from other types of research. Longitudinal or panel
\emph{data} are repeated observations on several units (i.e., \(N\) or
\(i\) \textgreater{} 1), whereas time-series data are observations of
one unit over time -- a distinction that focuses on the amount of people
in the study (given repeated measures). Most organizational studies
collect data on more than one unit, therefore our discussion below
focuses on longitudinal research with panel data, or designs with \(N\)
\textgreater{} 1, \(t\) \(\geq\) 3, and the same construct(s) measured
on each \(i\) at each \(t\). That is, we focus on designs with repeated
measures across many people (units) where every variable is measured at
each time point.

Longitudinal applies to both short and long-term research. An experiment
with ten trials is longitudinal, as is a study spanning 10 years that
assesses its measures once every year. Longitudinal is not reserved for
\enquote{long-term} studies that last more than one year irrespective of
the frequency of their observations. Rather, certain processes unfold
over short time horizons (e.g., decision-making on simple tasks, swift
action teams; Wildman et al., 2012) whereas other psychological
phenomena unfold over long time horizons (e.g., the development of a
shared organizational culture; Mitchell \& James, 2001), so the
informativeness of a particular study depends on its rationale, research
design, analytical work, and effective interpretation of results -- as
with any study. Short and long time horizons both offer valuable
insights.

\hypertarget{framework-for-longitudinal-inferences}{%
\section{Framework for Longitudinal
Inferences}\label{framework-for-longitudinal-inferences}}

We use three inference categories to partition our discussion, including
trends, relationships, and dynamics. Briefly, longitudinal inferences
focusing on trends assess whether trajectories follow a predictable
pattern or whether trajectories differ between-units; longitudinal
inferences focusing on relationships between constructs assess the
between-unit relationship among one or more constructs; longitudinal
inferences focusing on dynamics in constructs assess how one or more
constructs move through time as functions of themselves and each other
and emphasize how the past constrains the future. Each category comes
with box-and-arrow model heuristics\footnote{Note that statistical
  models differ from the term, \enquote{model heuristic.} A model
  heuristic is a visual representation only, whereas a statistical model
  is characterized by a formula explaining the data and assumptions on
  the errors, and the parameters of statistical models are estimated
  using an estimation technique. In this paper, we never use the term,
  \enquote{model} without pairing it either with \enquote{statistical}
  or \enquote{heuristic} -- the two differ substantially.} that
represent the broad inferences, research questions to orient the reader
as to what the sub-inferences capture (i.e., inferences are the answers
to the research questions that we present), and a discussion of
statistical models.

Although we use box-and-arrow diagrams throughout to represent the broad
inferences, we also graph a few of the more challenging inferences with
mock data -- some of the inferences in the trend and relationships
sections are difficult to grasp without seeing them in data form. Keep
in mind, however, that data are always messy. It is rare to find data in
which the inferences present themselves simply by plotting -- althought
it is certainly possible. We use these \enquote{data plots} to clearly
convey what the inferences mean, but be aware that field data are often
noisy.

Finally, despite pointing researchers to statistical models, our paper
puts a majority of its emphasis on inferences, therefore researchers
need to be sure that they appreciate all of the nuance before applying a
recommended statistical model. Numerous issues arise when modeling
longitudinal data and the statistical models differ in how they handle
these issues, the assumptions they make, and the data format they
require. We do not speak directly to those issues here, but we refer
readers to a number of informative references for each statistical
model.

\begin{figure}

{\centering \includegraphics[width=4.66in]{figures/dynamics/framework} 

}

\caption{Common inference categories with models applied to longitudinal data.\label{framework_figure}}\label{fig:unnamed-chunk-6}
\end{figure}

\hypertarget{trend}{%
\section{Trend}\label{trend}}

Made popular in the organizational literature by Bliese and Ployhart
(2002) and Chan (1998), trend inferences represent a class of thinking
in which researchers create an index of time and relate it to their
response variable to understand the trajectory of the dependent
variable. The first panel of Figure \ref{framework_figure} shows a
box-and-arrow model heuristic in which time is related to an outcome,
\(y\), and ultimately the analyst is interested in a variety of
questions about trend and its correlates. Trend inferences have two
components: trend itself and level. For clarity, we discuss them
separately.

\hypertarget{component-1---trend}{%
\subsubsection{Component 1 - Trend}\label{component-1---trend}}

Does affect, in general, increase or decrease across time, or is its
trajectory relatively flat? Does trainee skill generally increase over
the training session? These are questions about trend, and these first
two are specifically about linear trend. It is also possible to explore
how variables bend or curve across time. Do newcomer perceptions of
climate increase and then plateau over time? Does the response time of a
medical team decrease with each successive case but then remain stable
once the team can no longer improve their coordination? These latter
questions concern curvilinear trajectories.

Trend has to do with the systematic direction or global shape of a
trajectory across time. If we put a variable on the \(y\)-axis and plot
its values against time on the \(x\)-axis, do the values display a
stable temporal pattern? It can be thought of as the coarse-grained
direction of a trajectory. A positive trend indicates that, on average
across units, we expect the variable to increase over time and a
negative trend indicates that we expect the variable to decrease over
time. Our first trend research question, therefore, concerns the shape
of the trajectory.

\begin{quote}
\begin{quote}
\textbf{Research Question 1:} On average across units, is there a
positive/negative/curvilinear trend?
\end{quote}
\end{quote}

Many research questions and inferences begin with the average pattern
(or relationship) and then move to variability, the same applies here.
After learning about the average trend across the sample, researchers
then focus on trend variability. How much consistency is there in the
trend pattern? Do all trainees develop greater skill across time? Is
there variability in the trend of helping behaviors, or
counterproductive work behaviors over time?

\begin{quote}
\begin{quote}
\textbf{Research Question 2:} Does trend differ across units?
\end{quote}
\end{quote}

Research questions one and two concern one variable, but they can also
be iterated across all observed variables. For example, we might
discover that -- on average across units -- affect and performance
trends both decrease, but there is greater variability across units in
the affect trend. Or we might learn that affect has a negative trend
while performance has a positive trend.

Correlating these trends between-units is the next inference.
Correlating indicates co-occuring patterns, where a large, positive,
between-unit correlation between affect and performance trends indicates
that people with a positive affect trend (usually) have a positive
performance trend and people with a negative affect trend (usually) have
a negative performance trend.

Figure \ref{trend_correlation} shows the inuition behind this inference
with a set of graphs. In Panel A, we plot affect and performance across
time for three individuals. Affect goes up while performance goes down
for person one, this pattern is reversed for person two, and person
three reports trendless affect and performance (i.e., zero trend). We
use different colors to label the trends for each person. The affect and
performance trends for person one are labeled with red lines, whereas
person two has green lines and person three has blue lines.

Panel B then maps those pairings onto a scatterplot that demonstrates
the between-unit relationship among affect and performance trends. For
example, person one has a positive affect trend and a negative
performance trend, so their value in Panel B goes on the bottom right,
whereas person two has the opposite pattern and therefore is placed on
the top left (where the performance trend is positive and the affect
trend is negative). Producing this bottom scatter plot tells us that the
between-unit association among affect and performance trends is
negative. That is, people with a positive affect trend are expected to
have a negative performance trend, people with a negative affect trend
are expected to have a positive performance trend, and people with an
affect trend of zero are expected to have a performance trend of zero.

\begin{center}

------------

Insert Figure \ref{trend_correlation} about here

------------

\end{center}

\begin{quote}
\begin{quote}
\textbf{Research Question 3:} What is the between-unit correlation among
two trends?
\end{quote}
\end{quote}

The final trend inference is about identifying covariates or predictors
of trend. Does gender predict depletion trends? Does the trend in unit
climate covary with between-unit differences in leader quality?

Figure \ref{trend_covariate} demonstrates the inference in a plot. We
graph affect across time for six employees that differ by job type. The
first three individuals work in research and development, whereas the
final three work in sales. Affect trajectories tend to decrease over
time for employees in research and development, whereas affect
trajectories tend to increase for those in sales. An individual's job
type, then, gives us a clue to their likely affect trend -- said
formally, job type covaries with affect trend, such that we expect
individuals in sales to have positive affect trends and individuals in
research and development to have negative affect trends. The expected
trends are plotted as the thick blue lines.

\begin{center}

------------

Insert Figure \ref{trend_covariate} about here

------------

\end{center}

\begin{quote}
\begin{quote}
\textbf{Research Question 4:} What is the betweeen-unit correlation
among trend and a covariate?
\end{quote}
\end{quote}

Note the difference between research questions three and four. Both are
between unit, but three is about co-occuring trend patterns whereas four
is about the relationship between trend and a covariate/predictor. With
respect to our examples, inference three (i.e., the answer to research
question three) says, on average, if an individual has a positive affect
trend then we expect her to have a negative performance trend. Inference
four says, on average, if an individual is in research and development
then we expect him to have a negative affect trend.

\hypertarget{component-2---level}{%
\subsubsection{Component 2 - Level}\label{component-2---level}}

Researchers that explore trend also assess its predicted value at a
given time \(t\), and this second component is called level. Level is
almost always evaluated at the first or last observed time point --
e.g., What is the predicted level of the trainee skill trend, on average
across units, at the beginning of a training session? On average across
units, what is the expected level of the unit climate trend at the end
of a two-week socialization process? -- although researchers are free to
asssess level at any \(t\).

\begin{quote}
\begin{quote}
\textbf{Research Question 5:} On average across units, what is the
expected level of the \(y\) trend at time \(t\)?
\end{quote}
\end{quote}

After exploring the average (across units) trend level, we then move to
its variability. Trend lines have a beginning (or end) point, how
consistent do we expect that point to be across the sample? Is there
variability in affect trend starting level? Are there large between-unit
differences in the expected level of the performance trend at the last
time point?

\begin{quote}
\begin{quote}
\textbf{Research Question 6:} Is there variability across units in the
expected level of the \(y\) trend at time \(t\)?
\end{quote}
\end{quote}

It is also possible to assess between-unit correlations among level and
(a) trend in the same variable or (b) level or (c) trend in a different
variable. First, consider a relationship among level and trend in the
same variable. On average across units, do people with low initial skill
show positive skill trends whereas people with high initial skill show
negative skill trends? Do organizations with high initial CWBs, on
average across units, tend to have negative CWB trends?

\begin{quote}
\begin{quote}
\textbf{Research Question 7:} What is the between-unit correlation
between trend and level in \(y\)?
\end{quote}
\end{quote}

Second, consider a between-unit correlation between level in one
variable and level in another. On average across units, do people with
low initial performance also have low initial depletion (based on the
initial levels predicted by the performance and depletion trends)? Are
organizations with high initial turnover also expected, on average
across units, to have high burnout (based on the initial levels
predicted by the turnover and burnout trends)?

\begin{quote}
\begin{quote}
\textbf{Research Question 8:} What is the between-unit correlation
between level of the \(x\) trend and level of the \(y\) trend at \(t\)?
\end{quote}
\end{quote}

Finally, researchers are free to mix the inferences above and assess
whether levels in one variable covary with trend in another. Are people
with high initial voice (predicted by the voice trend) expected to have
negative satisfaction trends?

\begin{quote}
\begin{quote}
\textbf{Research Question 9:} What is the between-unit correlation
between the level of the \(x\) trend at time \(t\) and the trend in
\(y\)?
\end{quote}
\end{quote}

A note on phrasing. The inferences we explored in this section have to
do with correlating levels and trends, where a statement like,
\enquote{affect and performance trends covary between-units, such that
people with a negative affect trend have a positive performance trend}
is appropriate. There is a less precise phrase that is easy to fall into
-- and we have seen it used in our literature. Sometimes, researchers
will correlate trends and then state, \enquote{when affect decreases
performance goes up.} We urge researchers to avoid this second statement
because it is not clear if it refers to a static relationship about
trends or a dynamic statement about how trajectories move across time.
That is, the phrase \enquote{when affect decreases performance goes up}
could refer to between-unit correlated trends, but it could also mean
something like, \enquote{when affect decreases performance immediately
or subsequently goes up.} This second statement is far different and it
should not be used when an analysis only correlates trends or evokes
predictors of trend. Again, we urge researchers to phrase their
inferences as we show here.

\hypertarget{statistical-models-for-trend}{%
\subsection{Statistical Models for
Trend}\label{statistical-models-for-trend}}

Currently, the dominant method for analyzing longitudinal data with
respect to trend inferences in the organizational sciences is growth
curve modeling (GCM; Braun, Kuljanin, \& DeShon, 2013; Kuljanin et al.,
2011a). Broad theoretical discussions of growth are in Pitariu and
Ployhart (2010) and Ployhart and Vandenberg (2010) (keep in mind that
they call growth \enquote{change}), whereas Bliese and Ployhart (2002)
describe actual growth curve analysis. Growth curves are a core topic in
developmental psychology, so there are many articles and textbooks to
read from their field. See Grimm, Ram, and Estabrook (2016) and Singer,
Willett, and Willett (2003) for two great textbooks on growth curve
modeling and McArdle and Epstein (1987) for an empirical discussion. Two
straight-forward empirical examples from our own field include Dunford
et al. (2012) and Hülsheger (2016).

GCM is the longitudinal application of the more general statistical
technique, random coefficient modeling (RCM; e.g, Hierarchical Linear
Modeling; Latent Growth Modeling; Bollen \& Curran, 2006; Raudenbush \&
Bryk, 2002; Singer et al., 2003). GCM (and RCM) can be applied through
either a regression-based (e.g., Singer et al., 2003) or structural
equation modeling-based (SEM; e.g., Bollen \& Curran, 2006) approach. A
complete discussion of these two approaches is beyond the scope of this
paper; rather, this paper only focuses on the regression-based approach
of GCM. All models presented have an equivalent representation within
the SEM framework and can achieve identical inferences.

GCM models the dependent variable -- performance, for example -- as a
result of predictors at multiple levels of analysis. Level one
predictors vary at the same level as the depenent variable, meaning that
if individual performance is the outcome of interest then level one
predictors might include individual goal-striving and individual
cognitive ability. If, on the other hand, the outcome is organizational
performance then level one predictors might include organizational
climate or culture. Level two predictors occur at higher units of
analysis -- team cohesion if the dependent variable is individual
performance or national culture if the dependent variable is
organizational performance. Variables at any level can enter into the
statistical model either as fixed or random. Fixed predictors estimate
only the average relationship across all units, whereas random
predictors estimate not only the average IV-DV relationship across units
but also estimate the degree of between-unit variability in the
relationship.

The most basic growth model is the unconditional means model (UMM).
Using notation from Singer et al. (2003), this statistical model is
specified as

\begin{align}
\label{UMM}
Y_{ij} &= \pi_{0i} + \varepsilon_{ij} \\
\pi_{0i} &= \gamma_{00} + \zeta_{0i}
\end{align}

\noindent where \(\varepsilon\) \textasciitilde{}
\(N(0, \sigma_{\varepsilon}^2)\) and \(\zeta_{0i}\) \textasciitilde{}
\(N(0, \sigma_{0}^2)\), \(Y_{ij}\) is the dependent variable measured
for person \(i\) at time \(j\), \(\pi_{0i}\) is the mean of \(Y\) for
individual \(i\), \(\gamma_{00}\) is the mean of \(Y\) across everyone
in the population, \(\varepsilon_{ij}\) is the residual for individual
\(i\) on occasion \(j\), \(\sigma_{\varepsilon}^2\) is the pooled
within-person variance of each individual's data around his or her mean,
\(\zeta_{0i}\) is the random effect for individual \(i\) (i.e.,
deviation of the person-specific mean from the grand mean), and
\(\sigma_0^2\) is the random effect variance.

In words with individual performance as an example, this UMM says that
performance for Rachel at any time is a function of her across time
individual performance mean and error (equation one). Moreover, Rachel's
individual performance mean is a function of the population performance
mean (i.e., the mean of everyone's individual performance) and error
(equation two). What this statistical model embodies is that (1) a
reasonable prediction for Rachel's performance given no other
information, such as other predictors like goal-striving or cognitive
ability, is the mean of individual performance and (2) there are
between-person differences in individual performance.

The initial UMM model is typically used to calculate the intraclass
correlation coefficient (ICC(1)) which estimates the proportion of total
variance attributed to between-unit differences. Researchers, for
example, would conduct an initial UMM on individual performance and then
state, perhaps, that 57\% of the total variance resides between
individuals. It is also possible to conduct a \(\chi^2\) test to assess
whether the estimated between-person variance differs from zero. Both
results are used to argue that it is reasonable to move forward with
more complicated statistical analyses, to include additional predictors
that may explain (in a statistical sense) the observed between-person
variability.

Assuming differences across units exist, it is then recommended to
conduct the unconditional linear growth model (ULGM). The ULGM regresses
the dependent variable on a fixed linear time variable while allowing
variability across units in intercepts. The regression weight on the
time variable (ie., slope) models the average trend across units and is
used to answer RQ1. It is then common to allow the Time-DV relationship
to vary across units by entering the time variable as a level one random
predictor, as shown in the equation below.

\begin{align}
\label{ULGM}
Y_{ij} &= \pi_{0i} + \pi_{i1}Time_{ij} + \varepsilon_{ij} \\
\pi_{0i} &= \gamma_{00} + \zeta_{0i} \\
\pi_{i1} &= \gamma_{10} + \zeta_{1i}
\end{align}

\noindent \noindent where \(\varepsilon\) \textasciitilde{}
\(N(0, \sigma_{\varepsilon}^2)\) and
\(\begin{bmatrix} \zeta_{0i} \\ \zeta_{1i} \end{bmatrix}\)
\textasciitilde{}
\(N\bigl(\begin{bmatrix} 0 \\ 0 \end{bmatrix}, \begin{bmatrix} \sigma_{0}^2 & \sigma_{10} \\ \sigma_{10} & \sigma_{1}^2\end{bmatrix}\bigr)\),
\(\pi_{0i}\) is now the initial status (i.e., intercept) of \(Y\) for
individual \(i\), \(\gamma_{00}\) is the average initial status of \(Y\)
across everyone in the population, \(\pi_{1i}\) is the rate of change
(i.e., slope) of \(Y\) for individual \(i\), \(\gamma_{10}\) is the
average rate of change of \(Y\) across everyone in the population,
\(\sigma_{\varepsilon}^2\) is the pooled variance of each individuals'
data around his or her linear change trajectory, \(\zeta_{0i}\) is the
intercept random effect for individual \(i\), \(\sigma_{0}^2\) is the
variance of intercept random effects, \(\zeta_{1i}\) is the slope random
effect for individual \(i\), \(\sigma_1^2\) is the variance of the slope
random effects, \(\sigma_{10}\) is the population covariance between
intercepts and slopes, and all other terms are defined above.

In words with individual performance as example, this ULGM says that
Rachel's performance is a function of her initial level of performance
(which is a function of the population initial performance level and
error) and time (equation 3). Time, therefore, can be thought of as a
predictor in the case of the ULGM which makes it an inherently static,
as opposed to dynamic, statistical model (Voelkle \& Oud, 2015) and
emphasizes description rather than explanation because time cannot be a
true underlying cause (Pitariu \& Ployhart, 2010). We moved beyond the
unconditional means model to \enquote{explain} more variation in
Rachel's performance by including additional predictors. In the case of
the ULGM, our additional predictor is time and the estimate of the
coefficient relating it to the outcome describes both the expected trend
and whether there are between-unit differences in trend in the sample
(research questions one and two). Time is a level one predictor because
it varies on the same level as the outcome (the individual level) and it
is incorporated in the statistical model as a random effect (equation
5).

To review, we first modeled individual performance as a function of
across time individual performance means (which, themselves, were
functions of the population individual performance). That basic
statistical model was a UMM and we turned it into an ULGM by
incorporating time as a predictor. Once time enters the equation, we
update our view of performance and it becomes a function of initial
performance level and time, meaning that one-unit increases in time are
seen as relating to increases or decreases in performance. Those
increases and decreases across time, in aggregate, form trend. In
practice, the statistical model returns one number for the estimate of
the coefficient relating time to the outcome and it describes the
expected between-person performance trend.

Understanding the basic ULGM with time as a random level one predictor
allows researchers to explore, with simple extensions such as
incorporating additional predictors or modeling two or more variables as
outcomes (multivariate systems), any number of further inferences.
Researchers can enter additional multiples of time as predictors --
e.g., include \(Time^2\) and/or \(Time^3\) in equation 3 -- to determine
whether trajectories are curvilinear or follow other temporal patterns.
Researchers can also enter additional level one or two substantive
predictors to determine whether there are covariates of trend. Consider,
again, the example in Figure 3 which plots affect trends that differ by
job type. Affect is the outcome that is regressed on time, forming the
underlying (descriptive) ULGM. Entering job type as a random, level two
predictor returns a coefficient that describes whether the expected
affect trend differs according to this additional predictor.
Statistically, the model estimates whether higher values on the level
two predictor relate to stronger IV-DV relationships. In the case of
growth models, time is the IV so \enquote{stronger IV-DV} relationships
means different trend patterns. The level two predictor therefore
estimates whether higher values -- or in the case of job type, different
types of jobs -- demonstrate different trend (RQ4). Beyond incorporating
more predictors of a single outcome, researchers can also model multiple
outcomes with simultaneous ULGMs. Consider two independent ULGMs, one
with individual performance regressed on time and another with
individual OCBs regressed on time. All inferences, research questions,
and statitical models described above can be explored independently with
these two outcomes. Typically, however, when random effects are
incorporated covariances among all random effect variables are
estimated, meaning that the two outcomes in the multivariate system are
no longer viewed as independent. The covariance estimate between the
slope term for performance and the slope term for OCBs, in this example,
are used to answer research question three.

\hypertarget{statistical-models-for-level}{%
\subsection{Statistical Models for
Level}\label{statistical-models-for-level}}

Once time is included in the statistical model (e.g., the ULGM), the
intercept value represents the level of the DV at the time point coded 0
(typically the first or last time point). The intercept value is almost
always modeled as random whereby the analysis will return a mean
estimate which tells you the average level across units (answering RQ5),
and it will also return a variance estimate that indicates variability
in level between units. As such, the variance component on the intercept
term determines whether there is significant between-unit variability in
the level of the DV when time equals zero, answering RQ6. As previously
stated, it is common to estimate covariance among random predictors,
therefore the covariance between the intercept and slope random effects
is used to determine whether units with higher (lower) initial values
exhibit stronger (weaker) growth, answering RQ7. Finally, it is also
possible to estimate covariances among the intercept and slope among
different variables in multivariate systems, answering RQ8 and RQ9.

\begin{figure}
\centering
\includegraphics{figures/unnamed-chunk-10-1.pdf}
\caption{\label{fig:unnamed-chunk-10}Between-unit correlation of trend in
affect and performance.\label{trend_correlation}}
\end{figure}

\begin{figure}
\centering
\includegraphics{figures/unnamed-chunk-11-1.pdf}
\caption{\label{fig:unnamed-chunk-11}Job type as a covariate of affect
trend.\label{trend_covariate}}
\end{figure}

\hypertarget{relationships}{%
\section{Relationships}\label{relationships}}

A relationships inference explores between-unit relationships over time.
The second panel of Figure \ref{framework_figure} shows a model
heuristic, where a predictor is concurrently related to a response
variable at each time point and the relationship is typically
constrained to equality or is averaged over time. Essentially, the
inference compiles single-moment between-unit correlations. For example,
we assess the between-unit correlation between, say, OCBs and depletion
at time one, again and times two and three, and then ultimately take the
average of each individual, between-unit correlation.

Questions about static relationships over time take the following forms.
What is the relationship between helping behaviors and incivility? Does
burnout predict turnover intention? Is unethical behavior related to
self-control?

Figure \ref{relation_tvc} shows the inuition of the inference with data.
Panel A plots affect and performance trajectories for three people. The
red circles in Panel A highlight each individual's affect and
performance values at time point six. Given that we have three people at
time point six, we can calculate a correlation between affect and
performance at that moment (granted, it is a small sample). The
calculated coefficient is then graphed in Panel B with another red
circle. At time point six, the between-unit (across people) correlation
among affect and performance is large and positive.

\begin{center}

------------

Insert Figure \ref{relation_tvc} about here

------------

\end{center}

Panel B also shows between-unit correlation coefficients for the rest of
the time points. Often these (between-unit) correlations are either
averaged over time or constrained to be equal. Note that when a
researcher uses a time-varying covariates, hierarchical linear,
random-coefficient, or multi-level model on longitudinal data to explore
concurrent relationships among one or more variables (and they are not
analyzing trend) they are making this inference.

\begin{quote}
\begin{quote}
\textbf{Research Question 1:} What is the average between-unit
relationship of \(x\) and \(y\)? (Typically constrained to be equal over
time or averaged over time).
\end{quote}
\end{quote}

The first relationships inference emphasizes the between-unit expected
average. As with the trend inferences, the next question is to examine
variability in that estimated relationship across the sample. How
consistent across the sample is the relationship between distractions
and fatigue? Is there variability in the relationship between emotions
and volunteering behaviors?

\begin{quote}
\begin{quote}
\textbf{Research Question 2:} What is the variability across units in
the between-unit relationship among \(x\) and \(y\)?
\end{quote}
\end{quote}

\hypertarget{statistical-models-for-relationships}{%
\subsection{Statistical Models for
Relationships}\label{statistical-models-for-relationships}}

Time-varying covariates (TVC) analysis is the workhorse behind
relationship inferences. TVC models are simply growth curve models that
include level 1 predictors (either fixed or random). The equation below
shows an ULGM ammended to include an additional level one predictor,
\(X\).

\begin{align}
\label{ULGM}
Y_{ij} &= \pi_{0i} + \pi_{i1}Time_{ij} + \pi_{i2}X_{ij} + \varepsilon_{ij} \\
\pi_{0i} &= \gamma_{00} + \zeta_{0i} \\
\pi_{i1} &= \gamma_{10} + \zeta_{1i} \\
\pi_{i2} &= \gamma_{20} + \zeta_{2i}
\end{align}

\noindent \noindent where \(X\) is the additional random predictor and
it is related to the outcome, \(Y\), through \(\pi_{i2}\). The average
relationship between \(X\) and \(Y\) across units is used to answer RQ1
whereas the variance component estimating the between-unit variaiblity
in the \(X\)-\(Y\) relationship is used to answer RQ2. It is important
to note that TVC analyses can either be conducted by building upon the
ULGM, as is presented here, or can be done by building directly upon the
UMM (e.g., Judge, Scott, \& Ilies, 2006). The difference is whether the
predictor time is included to control for growth in the DV. Typically,
if it is anticipated or observed that the DV exhibits a consistent
trajectory over time, then time is included and TVC models build from
the ULGM. Alternatively, if the DV is not expected or observed to
exhibit linear (or curvilinear) growth, then level one predictors are
added directly to the UMM. A complete discussion of TVC models is in
Schonfeld and Rindskopf (2007) and Finch, Bolin, and Kelley (2016) and
two relatively straight-forward empirical examples are in Barnes,
Schaubroeck, Huth, and Ghumman (2011) and Chi, Chang, and Huang (2015).

\begin{figure}
\centering
\includegraphics{figures/unnamed-chunk-14-1.pdf}
\caption{\label{fig:unnamed-chunk-14}Relating affect to performance across
units over time. The red circles demonstrate the between unit
correlation at time point six. A typical time-varying covariates model
constrains the correlation to be equivalent across time. Here, the
relationship is unconstrained at each time point.\label{relation_tvc}}
\end{figure}

\hypertarget{dynamics}{%
\section{Dynamics}\label{dynamics}}

Dynamics refers to a specific branch of mathematics, but the term is
used in different ways throughout our literature. It is used informally
to mean \enquote{change}, \enquote{fluctuating,} \enquote{volatile,}
\enquote{longitudinal,} or \enquote{over time} (among others), whereas
formal definitions are presented within certain contexts. Wang, Zhou,
and Zhang (2016) define a dynamic \emph{model} as a
\enquote{representation of a system that evolves over time. In
particular it describes how the system evolves from a given state at
time \emph{t} to another state at time \emph{t + 1} as governed by the
transition rules and potential external inputs} (p.~242). Vancouver,
Wang, and Li (2018) state that dynamic \emph{variables} \enquote{behave
as if they have memory; that is, their value at any one time depends
somewhat on their previous value} (p.~604). Finally, Monge (1990)
suggests that in dynamic \emph{analyses}, \enquote{it is essential to
know how variables depend upon their own past history} (p.~409). In this
section we discuss a number of inferences couched in the idea that the
past constrains future behavior.

Does performance relate to itself over time? Do current helping
behaviors depend on prior helping behaviors? Does unit climate
demonstrate self-similarity across time? Does income now predict income
in the future? These are questions about the relationship of a single
variable with itself over time -- does it predict itself at each
subsequent moment? Is it constrained by where it was in the past?

Panel A of Figure \ref{dynamics_figure} shows the concept with a
box-and-arrow model heuristic. \(y\) predicts itself across every moment
-- it has self-similarity and its value now is constrained by where it
was a moment ago. In our diagram, we show that \(y\) at time \(t\) is
related to \(y\) at time \(t + 1\). In other words, we posit that \(y\)
shows a lag-one relationship, where \(y\) is related to its future value
one time-step away. Researchers are of course free to suggest any lag
amount that they believe captures the actual relationship. Note that the
statistical term to capture self-similarity or memory is called
autoregression.

\begin{quote}
\begin{quote}
\textbf{Research Question 1:} On average across units, what is the
relationship of \(y\) to itself over time? (Autoregression)
\end{quote}
\end{quote}

\begin{center}

------------

Insert Figure \ref{dynamics_figure} about here

------------

\end{center}

As before, after exploring the expected average we turn to variability.
How consistent are the self-similarity relationships? Are there
between-unit differences in autoregression in, for example, employee
voice? Do we expect a large variance in the autoregression of helping
behaviors?

\begin{quote}
\begin{quote}
\textbf{Research Question 2:} What is the variability across units in
the expected autoregression of \(y\)?
\end{quote}
\end{quote}

The next inference is about relating a predictor to our response
variable while it still retains memory. Panel B of Figure
\ref{dynamics_figure} shows a box-and-arrow diagram: \(y\) is predicted
by concurrent values of \(x\) but it also retains self-similarity. This
model is therefore said to partial prior \(y\): it examines the
concurrent relationship between \(x\) and \(y\) while statistically
partialling values of \(y\) at \(t - 1\), or statistically accounting
for \(y\) at the prior moment.

Our literature has converged on calling this kind of relationship
\enquote{change} because it emphasizes the difference between \(y\) now
and where it was in the past (e.g., Lanaj et al., 2016; Rosen et al.,
2016). The association asks how current \(x\) relates to the difference
between \(y\) now and its immediately prior value. How does affect
relate to change in performance? Does depletion covary with change in
OCBs? Note that change can be construed from any prior time point
(baseline, \(t-1\), \(t-3\)); our literature commonly emphasizes lag-one
change.

\begin{quote}
\begin{quote}
\textbf{Research Question 3:} On average across units, what is the
relationship betweeen concurrent \(x\) and change in \(y\)?
\end{quote}
\end{quote}

The analyst is also free to assess variability in the expected change
relationship.

\begin{quote}
\begin{quote}
\textbf{Research Question 4:} What is the variability across units in
the expected change relationship between concurrent \(x\) and \(y\)?
\end{quote}
\end{quote}

Change relationships do not have to be concurrent. Panel C of Figure
\ref{dynamics_figure} shows concurrent relationships as we saw above but
it also includes lags from the predictor to the outcome. \(y\) retains
memory, but it is predicted by both concurrent and prior values of
\(x\). Typically, we call these cross-lag relationships.

Questions about lag-one change relationships take the following forms.
Does affect predict subsequent performance change? Do prior
counterproductive work behaviors relate to current incivility change?
Does metacognition predict subsequent exploratory behavior change? Of
course, researchers can also explore longer lags by relating predictors
to more distal outcomes.

\begin{quote}
\begin{quote}
\textbf{Research Question 5:} On average across units, what is the
cross-lag relationship between \(x\) and change in \(y\) at a different
point in time?
\end{quote}
\end{quote}

Again, typically researchers explore variability after assessing the
average estimate.

\begin{quote}
\begin{quote}
\textbf{Research Question 6:} What is the variability over units in the
expected cross-lag relationship of change?
\end{quote}
\end{quote}

\hypertarget{extensions}{%
\subsection{Extensions}\label{extensions}}

We described a simple set of inferences above, but the ideas generalize
to more complex dynamics as well. Often researchers are interested in
reciprocal relationships, where \(x\) influences subsequent \(y\), which
then goes back to influence \(x\) at the next time point. Said formally,
\(x_t\) influences \(y_{t+1}\), which then influences \(x_{t+2}\). Said
informally, current performance influences subsequent self-efficacy,
which then influences performance on the next trial. These inferences
are no different than what we saw above -- they are cross-lag
predictions -- all we did was add more of them. Panel D of Figure
\ref{dynamics_figure} shows reciprocal dynamics, in which both \(x\) and
\(y\) show self-similarity and cross-lag relationships with one another.

Researchers typically posit a sequence of single cross-lag predictions
when they are interested in reciprocal dynamics. For example, Hardy III,
Day, and Steele (2018) explored reciprocal relationships among
performance and motivation (self-efficacy, metacognition, and
exploratory behavior). Their hypotheses include, (1) prior self-efficacy
negatively relates to subsequent exploratory behavior and (2) prior
exploratory behavior positively relates to subsequent self-efficacy
(among others). These single inferences are used in aggregate to make
conclusions about reciprocal influence.

The dynamic inferences shown here also generalize to systems of
variables where a researcher posits self-similarity and cross-lag
predictions across many variables. There could be reciprocal dynamics
between a set of variables like performance, self-efficacy, and affect,
or a sequence of influence between dyadic exchanges, performance, and
team perceptions: perhaps initial dyadic exchanges influence subsequent
team perceptions, which later influence performance. Following the
performance change, the structure of the task updates and this initiates
new dyadic exchanges. Once a researcher grasps the foundational ideas
presented here he or she is free to explore any number of complex
relationships.

\hypertarget{statistical-models-for-dynamics}{%
\subsection{Statistical Models for
Dynamics}\label{statistical-models-for-dynamics}}

Much like the models presented for relationship inferences, one way to
view dynamic models is as extensions of the ULGM or UMM. The principle
addition for dynamic models is the inclusion of a lagged version of the
DV as a predictor (\(Y_{t-1}\)). The inclusion of \(Y_{t-1}\) controls
for prior observations of the DV when predicting current values,
essentially modeling the change in the DV from one time point to another
without relying on difference scores (e.g., Edwards \& Parry, 1993). As
such, the first research question is answered by evaluating the average
relationship between the DV and a prior version of itself as a level one
predictor. Similarly, once the autoregressive term is modeled as random,
evaluating the variance component answers RQ2 regarding whether the
autoregressive relationship differs across units. To answer the
subsequent research questions, the inclusion of an additional
substantive predictor, \(X_t\), is required. When \(X_t\) is modeled at
only the concurrent time point with the DV, then the \(X_t\)
\(\rightarrow\) \(Y_t\) relationship determines whether values of \(X\)
at a given time point relate to the change in \(Y\), addressing RQ3. The
variance component on \(X_t\) when it is modeled as a random level one
preditor determines whether the relationship varies across units,
answering RQ4. Finally, if the researcher is interested in determining
whether changes in the predictor, \(X\), relate to changes in the DV,
\(Y\), an additional level one predictor in included in the model that
represnts prior realizations of \(X\), \(X_{t-1}\). With the inclusion
of \(X_{t-1}\), the paramater on the predictor \(X_t\) now determines
whether changes in \(X\) relate to changes in \(Y\), answering RQ5
whereas the variance component on \(X_t\) determines whether significant
variability in the relationship exists (RQ6). There are many additional
dynamic models that can be estimated within the GCM framework. Wang et
al. (2016) review a variety of dynamic models and, although their paper
does not provide readers with specific code, it is an excellent resource
to become familiar with potential dynamic models.

\begin{figure}

{\centering \includegraphics[width=3.93in]{figures/dynamics/dall} 

}

\caption{Univariate and bivariate dynamics adapted from DeShon (2012). Panel A shows self-similarity or autoregression in $Y$ across time. Panel B shows concurrent $X$ predicting change in $Y$. Panel C shows lagged change relationships. Panel D shows reciprocal dynamics between $X$ and $Y$.\label{dynamics_figure}}\label{fig:unnamed-chunk-16}
\end{figure}

\hypertarget{discussion}{%
\section{Discussion}\label{discussion}}

There are many different patterns to explore with longitudinal data
structures. This paper, by unpacking between-unit patterns, mirrors the
common questions and inferences currently emphasized by organizational
scientists. What is the between-unit relationship among a set of
constructs (averaged over time)? What is the between-unit expected
trend? Are there between-unit differences in trend (also phrased as,
\enquote{between-unit differences in within-unit change})? We organized
these questions and inferences into a fundamental set, discussed what
they mean, and linked the inferences to appropriate statistical models.
Ultimately, researchers should now be able to understand the spectrum of
between-unit inferences that they can explore with rich, longitudinal
data.

Between-unit questions are common and useful, but an alternative lens to
asking questions and making inferences with repeated measures is to
focus on within-unit patterns. Within-unit inferences emphasize
fluctuations over time rather than across units. For example, Beal
(2015) notes that many of the psychological phenomenon in which we are
interested are \enquote{sequences of events and event reactions that
happen within each person's stream of experience} (p.~5). This is a
within-unit statement: it emphasizes how a construct moves through time
within a single individual.

Organizational scientists have become increasingly interested in
within-unit perspectives over the past decade. Dalal, Bhave, and Fiset
(2014) review theory and research on within-person job performance,
Grandey and Gabriel (2015) review emotional labor and differentiate a
variety of within-person perspectives, Park, Spitzmuller, and DeShon
(2013) present a team motivation model describing within-individual
resource allocation and within-team feedback, Vancouver, Weinhardt, and
Schmidt (2010) present a within-person model of multiple-goal pursuit,
Barnes (2012) describes recent within-person approaches to sleep and
organizational behavior, and Methot, Lepak, Shipp, and Boswell (2017)
present a within-person perspective of organizational citizenship
behaviors.

Within-unit perspectives have their own research questions and
inferences. For example, Ilies, Johnson, Judge, and Keeney (2011)
hypothesize that \enquote{interpersonal conflict at work immediately
influences employee's negative affect, such that employees will report
heightened negative affect after periods when they experience more
conflict, compared to periods when they experience less conflict}
(p.~3). There are many within-person inferences accumulating in our
literature, but they often apply a between-person model and are
dispersed among different content areas. An immediate next step for
research is to write the within-unit version of this paper, a paper that
organizes and explains within-unit inferences.

When researchers explore patterns in longitudinal data, regardless of
whether they emphasize between or within-unit inferences, there are
additional statistical complexities to consider that influence the
veracity of a researcher's conclusions. For example, consider a
researcher interested in inference one from the \enquote{relationships}
section of this paper. To explore it, she collects data on 400 subjects
across eight time points, applies a recommended statistical model, and
then evaluates the results and makes an inference about the underlying
process. Although she aligned her question with an appropriate
statistical model, there is an issue related to her data that she did
not assess. The longitudinal data that she collected may not contain the
statistical characteristics that merit her inference. She can ask
questions about its patterns, apply a statistical model to it and make
statements that are appropriate \emph{given only the statistical model
that she applied}, but we do not know if her inference is appropriate
\emph{given the statistical characteristics of the data that she applied
her model to}. Do the data merit her inference in the first place?

The statistical complexities that we discuss below include stationarity
and ergodicity. Stationarity and ergodicity are statistical
characteristics that can be assessed with longitudinal data, and we
discuss both below in the context of advocating for greater \(T\), for
researchers to collect more observations over time because statistical
models alone do not reveal stationarity or ergodicity if the analyst is
not meticulously looking for them. They require tests of their own and
the tests are facilitated by data structures with more time points.

Processes give rise to observed data and those observed data are
characterized by distributions and their moments. Stationarity is about
whether or not the statistical characteristics of a process remain
stable over time. When they do, the analyst has permission to use a
variety of regression-based techniques like those described in this
paper without additional concerns of faulty inferences. When
trajectories are non-stationary, however, then the inferences drawn from
regression-based techniques are often misguided (Granger \& Newbold,
1974). Full explanations of stationarity are in Kuljanin et al. (2011b),
Braun et al. (2013), Jebb, Tay, Wang, and Huang (2015), and Metcalfe and
Cowpertwait (2009), we draw attention to it here to emphasize that
studies with greater \(T\) have the ability to assess stationarity and
understand which statistical models are appropriate. Moreover, finding
evidence of (non)stationary is useful theoretical knowledge and needs to
take the foreground of studies that collect longitudinal data.

Ergodicity is another statistical characteristic of a process and it is
important because it determines whether or not researchers can
generalize inferences of inter-individual variability from tests of
between-unit differences to inferences of within-unit variability. To
see the dilemma, consider the following. First, the standard statistical
models in psychology and management, such as growth curves, multi-level
models, mixture modeling, ANOVA, and factor analysis all focus on
between-unit variation (Molenaar, 2004). Second, researchers using these
techniques run their computations on a sample drawn from a population
and then generalize their results back to the population, so (a) the
results live at the level of the population and (b) researchers assume
that the population (or sub population in mixture modeling) is
homogenous (Molenaar \& Campbell, 2009). These notions are fine on their
own, but often an additional assumption creeps in that is unlikely to
hold: because resuls live at the level of the population and because
researchers assume that the population is homogenous they often also
assume that the results apply to the individuals making up the
population (Molenaar, 2008b). In other words, they assume that the
results from a test of between-unit variation hold at the level of
within-individual variation.

When processes are ergodic, this implicit assumption holds: the results
of an analysis of between-unit differences generalize to within-unit
patterns and vice versa (Molenaar, 2007, 2008a). Researchers can
generalize with ergodic processes, they can use a multi-level model to
assess between-unit patterns and then make statements about
within-person relationships. But this generalization is rarely
appropriate. A Gaussian process is non-ergodic if it is non-stationarity
(e.g., it has time-varying trends) and/or heterogeneous across subjects
(subject-specific dynamics). Stated simply, a Gaussian process is
non-ergodic if it has trend and/or Susie's trajectory is different from
Bob's. If either is violated, which is often the case, then standard
analyses of between-subject differences (growth models, multi-level or
random-coefficient models, mixture models, ANOVA, factor analysis)
cannot be used to make within-person statements. In general,
within-person inferences need to come from unpooled, subject-specific
time-series data structures (Molenaar, 2009).

Collecting large samples across many time points allows researchers to
assess stationarity and ergodicity. Both are complex ideas and merit
entire papers of their own, but for now we urge researchers to start
focusing on both so that our field can begin to understand the
similarities and differences among between-unit and within-unit
relationships. Again, researchers must collect data across many time
points to do so.

Often, though, researchers have finite resources and must decide whether
to emphasize between-unit or within-unit patterns. Your data collection
should align with the inference that you are interested in. If you care
about between-unit patterns (as shown in this paper), focus on \(N\) --
collect data on many participants. If you care about within-unit
patterns, focus on \(T\) -- collect data across many time points. Large
samples across many time points of course gives researchers the ability
to explore both frameworks, but our field will need to recognize that a
small samples (e.g., five or fewer participants) measured across many
time points does allow a researcher to make within-person inferences (by
definition) and is useful. Given the resource constraints that come with
conducting research, we cannot shy away from few participants measured
across many time points as viable techniques to assessing within-person
relationships.

\newpage

\hypertarget{references}{%
\section{References}\label{references}}

\setlength{\parindent}{-0.5in}
\setlength{\leftskip}{0.5in}

\hypertarget{refs}{}
\leavevmode\hypertarget{ref-barnes2012working}{}%
Barnes, C. M. (2012). Working in our sleep: Sleep and self-regulation in
organizations. \emph{Organizational Psychology Review}, \emph{2}(3),
234--257.

\leavevmode\hypertarget{ref-barnes_lack_2011}{}%
Barnes, C. M., Schaubroeck, J., Huth, M., \& Ghumman, S. (2011). Lack of
sleep and unethical conduct. \emph{Organizational Behavior and Human
Decision Processes}, \emph{115}(2), 169--180.

\leavevmode\hypertarget{ref-beal_esm_2015}{}%
Beal, D. J. (2015). ESM 2.0: State of the art and future potential of
experience sampling methods in organizational research. \emph{Annu. Rev.
Organ. Psychol. Organ. Behav.}, \emph{2}(1), 383--407.

\leavevmode\hypertarget{ref-bliese_growth_2002}{}%
Bliese, P. D., \& Ployhart, R. E. (2002). Growth modeling using random
coefficient models: Model building, testing, and illustrations.
\emph{Organizational Research Methods}, \emph{5}(4), 362--387.

\leavevmode\hypertarget{ref-bollen2006latent}{}%
Bollen, K. A., \& Curran, P. J. (2006). \emph{Latent curve models: A
structural equation perspective} (Vol. 467). John Wiley \& Sons.

\leavevmode\hypertarget{ref-braun_spurious_2013}{}%
Braun, M. T., Kuljanin, G., \& DeShon, R. P. (2013). Spurious Results in
the Analysis of Longitudinal Data in Organizational Research.
\emph{Organizational Research Methods}, \emph{16}(2), 302--330.
doi:\href{https://doi.org/10.1177/1094428112469668}{10.1177/1094428112469668}

\leavevmode\hypertarget{ref-chan1998conceptualization}{}%
Chan, D. (1998). The conceptualization and analysis of change over time:
An integrative approach incorporating longitudinal mean and covariance
structures analysis (lmacs) and multiple indicator latent growth
modeling (mlgm). \emph{Organizational Research Methods}, \emph{1}(4),
421--483.

\leavevmode\hypertarget{ref-chi_can_2015}{}%
Chi, N.-W., Chang, H.-T., \& Huang, H.-L. (2015). Can personality traits
and daily positive mood buffer the harmful effects of daily negative
mood on task performance and service sabotage? A self-control
perspective. \emph{Organizational Behavior and Human Decision
Processes}, \emph{131}, 1--15.

\leavevmode\hypertarget{ref-dalal2014within}{}%
Dalal, R. S., Bhave, D. P., \& Fiset, J. (2014). Within-person
variability in job performance: A theoretical review and research
agenda. \emph{Journal of Management}, \emph{40}(5), 1396--1436.

\leavevmode\hypertarget{ref-deshon_multivariate_2012}{}%
DeShon, R. P. (2012). Multivariate dynamics in organizational science.
\emph{The Oxford Handbook of Organizational Psychology}, \emph{1},
117--142.

\leavevmode\hypertarget{ref-dunford_is_2012}{}%
Dunford, B. B., Shipp, A. J., Boss, R. W., Angermeier, I., \& Boss, A.
D. (2012). Is burnout static or dynamic? A career transition perspective
of employee burnout trajectories. \emph{Journal of Applied Psychology},
\emph{97}(3), 637--650.
doi:\href{https://doi.org/http://dx.doi.org.proxy2.cl.msu.edu/10.1037/a0027060}{http://dx.doi.org.proxy2.cl.msu.edu/10.1037/a0027060}

\leavevmode\hypertarget{ref-edwards1993use}{}%
Edwards, J. R., \& Parry, M. E. (1993). On the use of polynomial
regression equations as an alternative to difference scores in
organizational research. \emph{Academy of Management Journal},
\emph{36}(6), 1577--1613.

\leavevmode\hypertarget{ref-finch2016multilevel}{}%
Finch, W. H., Bolin, J. E., \& Kelley, K. (2016). \emph{Multilevel
modeling using r}. Crc Press.

\leavevmode\hypertarget{ref-grandey2015emotional}{}%
Grandey, A. A., \& Gabriel, A. S. (2015). Emotional labor at a
crossroads: Where do we go from here?

\leavevmode\hypertarget{ref-granger_spurious_1974}{}%
Granger, C. W., \& Newbold, P. (1974). Spurious regressions in
econometrics. \emph{Journal of Econometrics}, \emph{2}(2), 111--120.

\leavevmode\hypertarget{ref-grimm_growth_2016}{}%
Grimm, K. J., Ram, N., \& Estabrook, R. (2016). \emph{Growth modeling:
Structural equation and multilevel modeling approaches}. Guilford
Publications.

\leavevmode\hypertarget{ref-hardy_interrelationships_2018}{}%
Hardy, J. H., Day, E. A., \& Steele, L. M. (2018). Interrelationships
Among Self-Regulated Learning Processes: Toward a Dynamic Process-Based
Model of Self-Regulated Learning. \emph{Journal of Management},
0149206318780440.
doi:\href{https://doi.org/10.1177/0149206318780440}{10.1177/0149206318780440}

\leavevmode\hypertarget{ref-hardy2018}{}%
Hardy III, J. H., Day, E. A., \& Steele, L. M. (2018).
Interrelationships among self-regulated learning processes: Toward a
dynamic process-based model of self-regulated learning. \emph{Journal of
Management}, 0149206318780440.

\leavevmode\hypertarget{ref-hulsheger_dawn_2016}{}%
Hülsheger, U. R. (2016). From dawn till dusk: Shedding light on the
recovery process by investigating daily change patterns in fatigue.
\emph{Journal of Applied Psychology}, \emph{101}(6), 905--914.
doi:\href{https://doi.org/http://dx.doi.org.proxy2.cl.msu.edu/10.1037/apl0000104}{http://dx.doi.org.proxy2.cl.msu.edu/10.1037/apl0000104}

\leavevmode\hypertarget{ref-ilgen_computational_2000}{}%
Ilgen, D. R., \& Hulin, C. L. (2000). \emph{Computational modeling of
behavior in organizations: The third scientific discipline.} American
Psychological Association.

\leavevmode\hypertarget{ref-ilies2011within}{}%
Ilies, R., Johnson, M. D., Judge, T. A., \& Keeney, J. (2011). A
within-individual study of interpersonal conflict as a work stressor:
Dispositional and situational moderators. \emph{Journal of
Organizational Behavior}, \emph{32}(1), 44--64.

\leavevmode\hypertarget{ref-jebb2015time}{}%
Jebb, A. T., Tay, L., Wang, W., \& Huang, Q. (2015). Time series
analysis for psychological research: Examining and forecasting change.
\emph{Frontiers in Psychology}, \emph{6}, 727.

\leavevmode\hypertarget{ref-jones_baby_2016}{}%
Jones, K. P., King, E. B., Gilrane, V. L., McCausland, T. C., Cortina,
J. M., \& Grimm, K. J. (2016). The baby bump: Managing a dynamic stigma
over time. \emph{Journal of Management}, \emph{42}(6), 1530--1556.

\leavevmode\hypertarget{ref-judge2006hostility}{}%
Judge, T. A., Scott, B. A., \& Ilies, R. (2006). Hostility, job
attitudes, and workplace deviance: Test of a multilevel model.
\emph{Journal of Applied Psychology}, \emph{91}(1), 126.

\leavevmode\hypertarget{ref-judge_what_2014}{}%
Judge, T. A., Simon, L. S., Hurst, C., \& Kelley, K. (2014). What I
experienced yesterday is who I am today: Relationship of work
motivations and behaviors to within-individual variation in the
five-factor model of personality. \emph{Journal of Applied Psychology},
\emph{99}(2), 199.

\leavevmode\hypertarget{ref-kozlowski_advancing_2013}{}%
Kozlowski, S. W., Chao, G. T., Grand, J. A., Braun, M. T., \& Kuljanin,
G. (2013). Advancing multilevel research design: Capturing the dynamics
of emergence. \emph{Organizational Research Methods}, \emph{16}(4),
581--615.

\leavevmode\hypertarget{ref-kozlowski_capturing_2016}{}%
Kozlowski, S. W., Chao, G. T., Grand, J. A., Braun, M. T., \& Kuljanin,
G. (2016). Capturing the multilevel dynamics of emergence: Computational
modeling, simulation, and virtual experimentation. \emph{Organizational
Psychology Review}, \emph{6}(1), 3--33.

\leavevmode\hypertarget{ref-kuljanin2011cautionary}{}%
Kuljanin, G., Braun, M. T., \& DeShon, R. P. (2011a). A cautionary note
on modeling growth trends in longitudinal data. \emph{Psychological
Methods}, \emph{16}(3), 249--264.

\leavevmode\hypertarget{ref-kuljanin_cautionary_2011}{}%
Kuljanin, G., Braun, M. T., \& DeShon, R. P. (2011b). A cautionary note
on modeling growth trends in longitudinal data. \emph{Psychological
Methods}, \emph{16}(3), 249--264.
doi:\href{https://doi.org/http://dx.doi.org.proxy2.cl.msu.edu/10.1037/a0023348}{http://dx.doi.org.proxy2.cl.msu.edu/10.1037/a0023348}

\leavevmode\hypertarget{ref-lanaj_when_2016}{}%
Lanaj, K., Johnson, R. E., \& Wang, M. (2016). When lending a hand
depletes the will: The daily costs and benefits of helping.
\emph{Journal of Applied Psychology; Washington}, \emph{101}(8), 1097.
Retrieved from
\url{http://search.proquest.com/docview/1813203845?pq-origsite=summon}

\leavevmode\hypertarget{ref-mcardle_latent_1987}{}%
McArdle, J. J., \& Epstein, D. (1987). Latent Growth Curves within
Developmental Structural Equation Models. \emph{Child Development},
\emph{58}(1), 110--133.
doi:\href{https://doi.org/10.2307/1130295}{10.2307/1130295}

\leavevmode\hypertarget{ref-metcalfe2009introductory}{}%
Metcalfe, A. V., \& Cowpertwait, P. S. (2009). \emph{Introductory time
series with r}. New York, NY: Chapman; Hall.

\leavevmode\hypertarget{ref-methot2017good}{}%
Methot, J. R., Lepak, D., Shipp, A. J., \& Boswell, W. R. (2017). Good
citizen interrupted: Calibrating a temporal theory of citizenship
behavior. \emph{Academy of Management Review}, \emph{42}(1), 10--31.

\leavevmode\hypertarget{ref-mitchell_building_2001}{}%
Mitchell, T. R., \& James, L. R. (2001). Building better theory: Time
and the specification of when things happen. \emph{Academy of Management
Review}, \emph{26}(4), 530--547.

\leavevmode\hypertarget{ref-molenaar_manifesto_2004}{}%
Molenaar, P. C. (2004). A manifesto on psychology as idiographic
science: Bringing the person back into scientific psychology, this time
forever. \emph{Measurement}, \emph{2}(4), 201--218.

\leavevmode\hypertarget{ref-molenaar2007psychological}{}%
Molenaar, P. C. (2007). Psychological methodology will change profoundly
due to the necessity to focus on intra-individual variation.
\emph{Integrative Psychological and Behavioral Science}, \emph{41}(1),
35--40.

\leavevmode\hypertarget{ref-molenaar2008consequences}{}%
Molenaar, P. C. (2008a). Consequences of the ergodic theorems for
classical test theory, factor analysis, and the analysis of
developmental processes. \emph{Handbook of Cognitive Aging}, 90--104.

\leavevmode\hypertarget{ref-molenaar2008implications}{}%
Molenaar, P. C. (2008b). On the implications of the classical ergodic
theorems: Analysis of developmental processes has to focus on
intra-individual variation. \emph{Developmental Psychobiology: The
Journal of the International Society for Developmental Psychobiology},
\emph{50}(1), 60--69.

\leavevmode\hypertarget{ref-molenaar2009generalization}{}%
Molenaar, P. C. (2009). How generalization works through the single
case: A simple idiographic process analysis of an individual
psychotherapy. In S. Salvatore, J. Valsiner, S. Strout, \& J. Clegg
(Eds.), \emph{YIS: Yearbook of idiographic science} (Vol. 1, pp.
23--38). Rome, Italy: Firera.

\leavevmode\hypertarget{ref-molenaar2009new}{}%
Molenaar, P. C., \& Campbell, C. G. (2009). The new person-specific
paradigm in psychology. \emph{Current Directions in Psychological
Science}, \emph{18}(2), 112--117.

\leavevmode\hypertarget{ref-monge_theoretical_1990}{}%
Monge, P. R. (1990). Theoretical and analytical issues in studying
organizational processes. \emph{Organization Science}, \emph{1}(4),
406--430.

\leavevmode\hypertarget{ref-park2013advancing}{}%
Park, G., Spitzmuller, M., \& DeShon, R. P. (2013). Advancing our
understanding of team motivation: Integrating conceptual approaches and
content areas. \emph{Journal of Management}, \emph{39}(5), 1339--1379.

\leavevmode\hypertarget{ref-pitariu_explaining_2010}{}%
Pitariu, A. H., \& Ployhart, R. E. (2010). Explaining change: Theorizing
and testing dynamic mediated longitudinal relationships. \emph{Journal
of Management}, \emph{36}(2), 405--429.

\leavevmode\hypertarget{ref-ployhart_longitudinal_2010}{}%
Ployhart, R. E., \& Vandenberg, R. J. (2010). Longitudinal research: The
theory, design, and analysis of change. \emph{Journal of Management},
\emph{36}(1), 94--120.

\leavevmode\hypertarget{ref-raudenbush2002hierarchical}{}%
Raudenbush, S. W., \& Bryk, A. S. (2002). \emph{Hierarchical linear
models: Applications and data analysis methods} (Vol. 1). Sage.

\leavevmode\hypertarget{ref-rosen_who_2016}{}%
Rosen, C. C., Koopman, J., Gabriel, A. S., \& Johnson, R. E. (2016). Who
strikes back? A daily investigation of when and why incivility begets
incivility. \emph{Journal of Applied Psychology}, \emph{101}(11), 1620.

\leavevmode\hypertarget{ref-schonfeld2007hierarchical}{}%
Schonfeld, I. S., \& Rindskopf, D. (2007). Hierarchical linear modeling
in organizational research: Longitudinal data outside the context of
growth modeling. \emph{Organizational Research Methods}, \emph{10}(3),
417--429.

\leavevmode\hypertarget{ref-scott_multilevel_2011}{}%
Scott, B. A., \& Barnes, C. M. (2011). A multilevel field investigation
of emotional labor, affect, work withdrawal, and gender. \emph{Academy
of Management Journal}, \emph{54}(1), 116--136.

\leavevmode\hypertarget{ref-singer_applied_2003}{}%
Singer, J. D., Willett, J. B., \& Willett, J. B. (2003). \emph{Applied
longitudinal data analysis: Modeling change and event occurrence}.
Oxford university press.

\leavevmode\hypertarget{ref-vancouver_translating_2018}{}%
Vancouver, J. B., Wang, M., \& Li, X. (2018). Translating Informal
Theories Into Formal Theories: The Case of the Dynamic Computational
Model of the Integrated Model of Work Motivation. \emph{Organizational
Research Methods}, 109442811878030.
doi:\href{https://doi.org/10.1177/1094428118780308}{10.1177/1094428118780308}

\leavevmode\hypertarget{ref-vancouver2010formal}{}%
Vancouver, J. B., Weinhardt, J. M., \& Schmidt, A. M. (2010). A formal,
computational theory of multiple-goal pursuit: Integrating goal-choice
and goal-striving processes. \emph{Journal of Applied Psychology},
\emph{95}(6), 985.

\leavevmode\hypertarget{ref-voelkle2015relating}{}%
Voelkle, M. C., \& Oud, J. H. (2015). Relating latent change score and
continuous time models. \emph{Structural Equation Modeling: A
Multidisciplinary Journal}, \emph{22}(3), 366--381.

\leavevmode\hypertarget{ref-Wang2016}{}%
Wang, M., Zhou, L., \& Zhang, Z. (2016). Dynamic modeling. \emph{Annual
Review of Organizational Psychology and Organizational Behavior},
\emph{3}(1), 241--266.
doi:\href{https://doi.org/10.1146/annurev-orgpsych-041015-062553}{10.1146/annurev-orgpsych-041015-062553}

\leavevmode\hypertarget{ref-wildman2012trust}{}%
Wildman, J. L., Shuffler, M. L., Lazzara, E. H., Fiore, S. M., Burke, C.
S., Salas, E., \& Garven, S. (2012). Trust development in swift starting
action teams: A multilevel framework. \emph{Group \& Organization
Management}, \emph{37}(2), 137--170.


\end{document}
