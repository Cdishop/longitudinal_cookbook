\documentclass[english,,man]{apa6}
\usepackage{lmodern}
\usepackage{amssymb,amsmath}
\usepackage{ifxetex,ifluatex}
\usepackage{fixltx2e} % provides \textsubscript
\ifnum 0\ifxetex 1\fi\ifluatex 1\fi=0 % if pdftex
  \usepackage[T1]{fontenc}
  \usepackage[utf8]{inputenc}
\else % if luatex or xelatex
  \ifxetex
    \usepackage{mathspec}
  \else
    \usepackage{fontspec}
  \fi
  \defaultfontfeatures{Ligatures=TeX,Scale=MatchLowercase}
\fi
% use upquote if available, for straight quotes in verbatim environments
\IfFileExists{upquote.sty}{\usepackage{upquote}}{}
% use microtype if available
\IfFileExists{microtype.sty}{%
\usepackage{microtype}
\UseMicrotypeSet[protrusion]{basicmath} % disable protrusion for tt fonts
}{}
\usepackage{hyperref}
\hypersetup{unicode=true,
            pdftitle={Inferences With Longitudinal Data},
            pdfauthor={\ldots{}},
            pdfkeywords={\ldots{}.},
            pdfborder={0 0 0},
            breaklinks=true}
\urlstyle{same}  % don't use monospace font for urls
\ifnum 0\ifxetex 1\fi\ifluatex 1\fi=0 % if pdftex
  \usepackage[shorthands=off,main=english]{babel}
\else
  \usepackage{polyglossia}
  \setmainlanguage[]{english}
\fi
\usepackage{color}
\usepackage{fancyvrb}
\newcommand{\VerbBar}{|}
\newcommand{\VERB}{\Verb[commandchars=\\\{\}]}
\DefineVerbatimEnvironment{Highlighting}{Verbatim}{commandchars=\\\{\}}
% Add ',fontsize=\small' for more characters per line
\usepackage{framed}
\definecolor{shadecolor}{RGB}{248,248,248}
\newenvironment{Shaded}{\begin{snugshade}}{\end{snugshade}}
\newcommand{\AlertTok}[1]{\textcolor[rgb]{0.94,0.16,0.16}{#1}}
\newcommand{\AnnotationTok}[1]{\textcolor[rgb]{0.56,0.35,0.01}{\textbf{\textit{#1}}}}
\newcommand{\AttributeTok}[1]{\textcolor[rgb]{0.77,0.63,0.00}{#1}}
\newcommand{\BaseNTok}[1]{\textcolor[rgb]{0.00,0.00,0.81}{#1}}
\newcommand{\BuiltInTok}[1]{#1}
\newcommand{\CharTok}[1]{\textcolor[rgb]{0.31,0.60,0.02}{#1}}
\newcommand{\CommentTok}[1]{\textcolor[rgb]{0.56,0.35,0.01}{\textit{#1}}}
\newcommand{\CommentVarTok}[1]{\textcolor[rgb]{0.56,0.35,0.01}{\textbf{\textit{#1}}}}
\newcommand{\ConstantTok}[1]{\textcolor[rgb]{0.00,0.00,0.00}{#1}}
\newcommand{\ControlFlowTok}[1]{\textcolor[rgb]{0.13,0.29,0.53}{\textbf{#1}}}
\newcommand{\DataTypeTok}[1]{\textcolor[rgb]{0.13,0.29,0.53}{#1}}
\newcommand{\DecValTok}[1]{\textcolor[rgb]{0.00,0.00,0.81}{#1}}
\newcommand{\DocumentationTok}[1]{\textcolor[rgb]{0.56,0.35,0.01}{\textbf{\textit{#1}}}}
\newcommand{\ErrorTok}[1]{\textcolor[rgb]{0.64,0.00,0.00}{\textbf{#1}}}
\newcommand{\ExtensionTok}[1]{#1}
\newcommand{\FloatTok}[1]{\textcolor[rgb]{0.00,0.00,0.81}{#1}}
\newcommand{\FunctionTok}[1]{\textcolor[rgb]{0.00,0.00,0.00}{#1}}
\newcommand{\ImportTok}[1]{#1}
\newcommand{\InformationTok}[1]{\textcolor[rgb]{0.56,0.35,0.01}{\textbf{\textit{#1}}}}
\newcommand{\KeywordTok}[1]{\textcolor[rgb]{0.13,0.29,0.53}{\textbf{#1}}}
\newcommand{\NormalTok}[1]{#1}
\newcommand{\OperatorTok}[1]{\textcolor[rgb]{0.81,0.36,0.00}{\textbf{#1}}}
\newcommand{\OtherTok}[1]{\textcolor[rgb]{0.56,0.35,0.01}{#1}}
\newcommand{\PreprocessorTok}[1]{\textcolor[rgb]{0.56,0.35,0.01}{\textit{#1}}}
\newcommand{\RegionMarkerTok}[1]{#1}
\newcommand{\SpecialCharTok}[1]{\textcolor[rgb]{0.00,0.00,0.00}{#1}}
\newcommand{\SpecialStringTok}[1]{\textcolor[rgb]{0.31,0.60,0.02}{#1}}
\newcommand{\StringTok}[1]{\textcolor[rgb]{0.31,0.60,0.02}{#1}}
\newcommand{\VariableTok}[1]{\textcolor[rgb]{0.00,0.00,0.00}{#1}}
\newcommand{\VerbatimStringTok}[1]{\textcolor[rgb]{0.31,0.60,0.02}{#1}}
\newcommand{\WarningTok}[1]{\textcolor[rgb]{0.56,0.35,0.01}{\textbf{\textit{#1}}}}
\usepackage{graphicx,grffile}
\makeatletter
\def\maxwidth{\ifdim\Gin@nat@width>\linewidth\linewidth\else\Gin@nat@width\fi}
\def\maxheight{\ifdim\Gin@nat@height>\textheight\textheight\else\Gin@nat@height\fi}
\makeatother
% Scale images if necessary, so that they will not overflow the page
% margins by default, and it is still possible to overwrite the defaults
% using explicit options in \includegraphics[width, height, ...]{}
\setkeys{Gin}{width=\maxwidth,height=\maxheight,keepaspectratio}
\IfFileExists{parskip.sty}{%
\usepackage{parskip}
}{% else
\setlength{\parindent}{0pt}
\setlength{\parskip}{6pt plus 2pt minus 1pt}
}
\setlength{\emergencystretch}{3em}  % prevent overfull lines
\providecommand{\tightlist}{%
  \setlength{\itemsep}{0pt}\setlength{\parskip}{0pt}}
\setcounter{secnumdepth}{0}
% Redefines (sub)paragraphs to behave more like sections
\ifx\paragraph\undefined\else
\let\oldparagraph\paragraph
\renewcommand{\paragraph}[1]{\oldparagraph{#1}\mbox{}}
\fi
\ifx\subparagraph\undefined\else
\let\oldsubparagraph\subparagraph
\renewcommand{\subparagraph}[1]{\oldsubparagraph{#1}\mbox{}}
\fi

%%% Use protect on footnotes to avoid problems with footnotes in titles
\let\rmarkdownfootnote\footnote%
\def\footnote{\protect\rmarkdownfootnote}


  \title{Inferences With Longitudinal Data}
    \author{\ldots{}\textsuperscript{1}}
    \date{}
  
\shorttitle{LONGITUDINAL INFERENCES}
\affiliation{
\vspace{0.5cm}
\textsuperscript{1} ...}
\keywords{....\newline\indent Word count: 95}
\usepackage{csquotes}
\usepackage{upgreek}
\captionsetup{font=singlespacing,justification=justified}

\usepackage{longtable}
\usepackage{lscape}
\usepackage{multirow}
\usepackage{tabularx}
\usepackage[flushleft]{threeparttable}
\usepackage{threeparttablex}

\newenvironment{lltable}{\begin{landscape}\begin{center}\begin{ThreePartTable}}{\end{ThreePartTable}\end{center}\end{landscape}}

\makeatletter
\newcommand\LastLTentrywidth{1em}
\newlength\longtablewidth
\setlength{\longtablewidth}{1in}
\newcommand{\getlongtablewidth}{\begingroup \ifcsname LT@\roman{LT@tables}\endcsname \global\longtablewidth=0pt \renewcommand{\LT@entry}[2]{\global\advance\longtablewidth by ##2\relax\gdef\LastLTentrywidth{##2}}\@nameuse{LT@\roman{LT@tables}} \fi \endgroup}


\DeclareDelayedFloatFlavor{ThreePartTable}{table}
\DeclareDelayedFloatFlavor{lltable}{table}
\DeclareDelayedFloatFlavor*{longtable}{table}
\makeatletter
\renewcommand{\efloat@iwrite}[1]{\immediate\expandafter\protected@write\csname efloat@post#1\endcsname{}}
\makeatother
\usepackage{lineno}

\linenumbers

\authornote{\ldots{}.

Correspondence concerning this article should be addressed to \ldots{},
\ldots{}. E-mail: \ldots{}}

\abstract{
Begin here\ldots{}


}

\usepackage{amsthm}
\newtheorem{theorem}{Theorem}[section]
\newtheorem{lemma}{Lemma}[section]
\theoremstyle{definition}
\newtheorem{definition}{Definition}[section]
\newtheorem{corollary}{Corollary}[section]
\newtheorem{proposition}{Proposition}[section]
\theoremstyle{definition}
\newtheorem{example}{Example}[section]
\theoremstyle{definition}
\newtheorem{exercise}{Exercise}[section]
\theoremstyle{remark}
\newtheorem*{remark}{Remark}
\newtheorem*{solution}{Solution}
\begin{document}
\maketitle

Organizational phenomena unfold over time. They are processes that
develop, change, and evolve (Pitariu \& Ployhart, 2010) that create a
sequence of events within a person's stream of experience (Beal, 2015).
Moreover, organizations are systems with many connected parts, and
systems are inherently dynamic. Studying these systems and processes,
therefore, requires paying attention not to static snapshots of behavior
(Ilgen \& Hulin, 2000), but variables and relationships as they move
through time; doing so puts us in a better position to capture the
sequence, understand it, and can lead to new and interesting insights
(Kozlowski \& Bell, 2003).

This sentiment is reflected in our empirical literature, where repeated
assessments are now common. For instance, Jones et al. (2016) observed
the work attitudes of pregnant women in their second trimester every
week until they gave birth. Meier and Spector (2013) examined
counterproductive work behavior over five waves. Hardy, Day, and Steele
(2018) investigated self-regulation over 20 lab trials. Finally,
Johnson, Lanaj, and Barnes (2014) observed justice behavior and resource
depletion across 10 consecutive workdays.

Armed with repeated observations, there are then different research
questions that we can explore. Jones et al. (2016) ask about trend: they
want to determine if the trajectories among certain variables increase
or decrease over time. Johnson et al. (2014) about change: they are
interested in how changes in one variable relate to changes in another
across time. Hardy et al. (2018) inquire about dynamic relationships,
where prior values on one variable predict subsequent values on another,
and this second variable then goes back to predict the first at a later
point in time. Finally, Meier and Spector (2013) examine how effect
sizes change when they vary the time lag between their independent and
dependent variable.

Researchers then evoke statistical models that are determined by their
research questions. Meier and Spector (2013) present a sequence of path
models that test increasingly longer time lags. Hardy et al. (2018) and
Jones et al. (2016) employ bivariate cross-lagged latent growth curves,
an approach similar to the latent change model used by Ritter, Matthews,
Ford, and Henderson (2016) We also find complex hierarchical linear
models in many event-sampling studies (e.g., Koopman, Lanaj, \& Scott,
2016; Rosen, Koopman, Gabriel, \& Johnson, 2016).

The spine of an investigation, finally, is to interpret the model and
make an inference regarding the original question. Jones et al. (2016)
infer negative slopes for concealing behaviors and positive slopes for
revealing behaviors. Johnson et al. (2014) state that justice behaviors
fluctuate day to day and predict changes in depletion. Hardy et al.
(2018) find support for dynamic relationships between self-efficacy,
metacognition, and exploratory behaviors. Finally, Meier and Spector
(2013) suggest that the effects of work stressors on counterproductive
work behaviors are not substantially different across different time
lags.

None of these inferences perfectly discovers the data generating
mechanism. Rather, each asks an interesting and important question about
how DVs relate to IVs. Only with lots of asking about lots of different
patterns of relationships across the variables could we piece together
one (of many) possible representation(s) of the data generating process
-- hopefully having a good theory to guide the way.

We want to link inferences to models in this paper so that researchers
know which of the many models they can use when they are interested in
one of the many possible inferences in a longitudinal investigation. As
should be clear to anyone reading our literature, there is great
excitement for the utility of longitudinal studies; they can pose
interesting questions and discover patterns that would otherwise be
impossible to capture in a static investigation. We bring attention to
the span of questions available so that researchers can fully appreciate
and take advantage of their data. Although the inferences concern
trajectories or relationships over time, their small differences have
large implications for what we take away from them -- what we ultimately
conclude. Moreoever, there are many inferences, many models, and
different models can be used to understand or explore the same
inference. In this paper, we provide readers with a specific model for
each inference so that they can be sure that the model they evoke is
appropriate for the research question that they are interested in. In
summary, this paper exposes researchers to the span of inferences they
may investigate when they collect longitudinal data, links those
inferences to models, and parses some of the modeling literature that
may be difficult to consume for researchers with only graduate level
training in statistics.

Below, we do these things.

\hypertarget{longitudinal-definitions}{%
\section{Longitudinal Definitions}\label{longitudinal-definitions}}

This paper is exclusively devoted to the inferences we make with
repeated observations, so we begin by identifying a few labels and
definitions. Authors typically identify a \enquote{longitudinal} study
by making a contrast with respect to either a) research designs or b)
data structures. Longitudinal \emph{research} is different from
cross-sectional research because longitudinal designs entail three or
more repeated observations (Ployhart \& Vandenberg, 2010). We therefore
emphasize differences on the number of observations when we distinguish
longitudinal from other types of research. Longitudinal \emph{data} are
repeated observations on several units (i.e., \(N\) or \(i\)
\textgreater{} 1), whereas panel data are observations of one unit over
time -- a distinction that focuses on the amount of people in our study
(given repeated measures). Most organizational studies collect data on
more than one unit, therefore our discussion below focuses on
longitudinal research with longitudinal data, or designs with \(N\)
\textgreater{} 1, \(t\) \textgreater{}= 3, and the same construct(s)
measured on (potentially) each \(i\) at (potentially) each \(t\).

\hypertarget{framework}{%
\subsection{Framework}\label{framework}}

Relationships. Growth. Change. Dynamics. These are umbrella research
foci, each has its own sub-inferences and models.

\hypertarget{relationships}{%
\section{Relationships}\label{relationships}}

General discussion.

\hypertarget{inference-1}{%
\subsection{Inference 1}\label{inference-1}}

A stable \(x\) relates to \(y\).

\hypertarget{model}{%
\subsubsection{Model}\label{model}}

.

\begin{Shaded}
\begin{Highlighting}[]
\StringTok{'}
\StringTok{perf.1 ~ b1*gender}
\StringTok{perf.2 ~ b1*gender}
\StringTok{perf.3 ~ b1*gender}
\StringTok{perf.4 ~ b1*gender}

\StringTok{'}
\end{Highlighting}
\end{Shaded}

\hypertarget{inference-2}{%
\subsection{Inference 2}\label{inference-2}}

A fluctuating \(x\) relates to \(y\).

\hypertarget{model-1}{%
\subsubsection{Model}\label{model-1}}

.

\begin{Shaded}
\begin{Highlighting}[]
\StringTok{'}
\StringTok{perf.1 ~ b1*affect.1}
\StringTok{perf.2 ~ b1*affect.2}
\StringTok{perf.3 ~ b1*affect.3}
\StringTok{perf.4 ~ b1*affect.4}

\StringTok{'}
\end{Highlighting}
\end{Shaded}

\hypertarget{growth}{%
\section{Growth}\label{growth}}

General discussion.

\hypertarget{inference-1-1}{%
\subsection{Inference 1}\label{inference-1-1}}

There is growth (positive or negative) in a given variable. Other terms:
trend, slope, some call this change; we won't.

\hypertarget{model-2}{%
\subsubsection{Model}\label{model-2}}

.

\begin{Shaded}
\begin{Highlighting}[]
\StringTok{'}
\StringTok{latent_perf_slope =~ 0*perf.1 + 1*perf.2 + 2*perf.3 + 3*perf.4}

\StringTok{'}
\end{Highlighting}
\end{Shaded}

\hypertarget{inference-2-1}{%
\subsection{Inference 2}\label{inference-2-1}}

There are inter-individual differences in growth.

\hypertarget{model-3}{%
\subsubsection{Model}\label{model-3}}

.

\begin{Shaded}
\begin{Highlighting}[]
\StringTok{'}
\StringTok{latent_perf_slope =~ 0*perf.1 + 1*perf.2 + 2*perf.3 + 3*perf.4}

\StringTok{'}
\end{Highlighting}
\end{Shaded}

\hypertarget{inference-3}{%
\subsection{Inference 3}\label{inference-3}}

There is a relationship between growth (slope) and level in a given
variable.

\hypertarget{model-4}{%
\subsubsection{Model}\label{model-4}}

.

\begin{Shaded}
\begin{Highlighting}[]
\StringTok{'}
\StringTok{latent_perf_level ~~ latent_perf_slope}

\StringTok{'}
\end{Highlighting}
\end{Shaded}

\hypertarget{inference-4}{%
\subsection{Inference 4}\label{inference-4}}

There is a relationship between a stable \(x\) and growth in \(y\).
There are inter-individual characteristics that relate to
inter-individual differences in slope.

\hypertarget{model-5}{%
\subsubsection{Model}\label{model-5}}

.

\begin{Shaded}
\begin{Highlighting}[]
\StringTok{'}
\StringTok{latent_perf_slope ~ b1*gender}

\StringTok{'}
\end{Highlighting}
\end{Shaded}

\noindent Could also do this for level.

\hypertarget{inference-5}{%
\subsection{Inference 5}\label{inference-5}}

There is a relationship between a fluctuating \(x\) and \(y\) after
partialling the growth in \(y\). Or, there is growth in \(y\) after
partialling the relationship between a fluctuating \(x\) and \(y\).

\hypertarget{model-6}{%
\subsubsection{Model}\label{model-6}}

.

\begin{Shaded}
\begin{Highlighting}[]
\StringTok{'}
\StringTok{latent_per_slope =~ 0*perf.1 + 1*perf.2 + 2*perf.3 + 3*perf.4}

\StringTok{perf.1 ~ b1*affect.1}
\StringTok{perf.2 ~ b1*affect.2}
\StringTok{perf.3 ~ b1*affect.3}
\StringTok{perf.4 ~ b1*affect.4}

\StringTok{'}
\end{Highlighting}
\end{Shaded}

\hypertarget{growth-2.0}{%
\section{Growth 2.0}\label{growth-2.0}}

Above, we examined growth in \(y\) and how it related to correlates or
predictors -- but those predictors/correlates were assumed to have no
growth. There is also a class of models for examining relationships
between two variables where both are assumed to grow.

\hypertarget{inference-1-2}{%
\subsection{Inference 1}\label{inference-1-2}}

There are correlated slopes among two growth curves.

\hypertarget{model-7}{%
\subsubsection{Model}\label{model-7}}

.

\begin{Shaded}
\begin{Highlighting}[]
\StringTok{'}

\StringTok{latent_perf_slope ~~ latent_affect_slope}

\StringTok{'}
\end{Highlighting}
\end{Shaded}

\hypertarget{change}{%
\section{Change}\label{change}}

General Discussion.

\hypertarget{inference-1-3}{%
\subsection{Inference 1}\label{inference-1-3}}

\(x\) is associated with a change in \(y\): an increase or decrease.

\hypertarget{model-8}{%
\subsubsection{Model}\label{model-8}}

.

\begin{Shaded}
\begin{Highlighting}[]
\StringTok{'}

\StringTok{perf.2 ~ b1*affect.2 + g1*perf.1}
\StringTok{perf.3 ~ b1*affect.3 + g1*perf.2}
\StringTok{perf.4 ~ b1*affect.4 + g1*perf.3}

\StringTok{'}
\end{Highlighting}
\end{Shaded}

\hypertarget{dynamics}{%
\section{Dynamics}\label{dynamics}}

General discussion.

\hypertarget{inference-1-4}{%
\subsection{Inference 1}\label{inference-1-4}}

There is autoregression in a variable, a relationship between prior and
future values.

\hypertarget{model-9}{%
\subsubsection{Model}\label{model-9}}

.

\begin{Shaded}
\begin{Highlighting}[]
\StringTok{'}
\StringTok{perf.2 ~ g1*perf.1}
\StringTok{perf.3 ~ g1*perf.2}
\StringTok{perf.4 ~ g1*perf.3}

\StringTok{'}
\end{Highlighting}
\end{Shaded}

\hypertarget{inference-2-2}{%
\subsection{Inference 2}\label{inference-2-2}}

There are cross-lag effects, where one variable relates to another at a
different point in time.

\hypertarget{model-10}{%
\subsubsection{Model}\label{model-10}}

.

\begin{Shaded}
\begin{Highlighting}[]
\StringTok{'}

\StringTok{perf.2 ~ b1*affect.1}
\StringTok{perf.3 ~ b1*affect.2}
\StringTok{perf.4 ~ b1*affect.3}

\StringTok{'}
\end{Highlighting}
\end{Shaded}

\hypertarget{inference-3-1}{%
\subsection{Inference 3}\label{inference-3-1}}

There is a reciprocal relationship, or feedback, where one variable
subsequently relates to another, and this second variable then relates
to the first at an even later point in time.

\hypertarget{model-11}{%
\subsubsection{Model}\label{model-11}}

.

\begin{Shaded}
\begin{Highlighting}[]
\StringTok{'}

\StringTok{perf.2 ~ b1*affect.1}
\StringTok{perf.3 ~ b1*affect.2}
\StringTok{perf.4 ~ b1*affect.3}

\StringTok{affect.2 ~ b2*perf.1}
\StringTok{affect.3 ~ b2*perf.2}
\StringTok{affect.4 ~ b2*perf.3}

\StringTok{'}
\end{Highlighting}
\end{Shaded}

\hypertarget{summary-list-of-inferences}{%
\section{Summary List of Inferences}\label{summary-list-of-inferences}}

\hypertarget{relationships-1}{%
\section{Relationships}\label{relationships-1}}

\hypertarget{inference-1-5}{%
\subsection{Inference 1}\label{inference-1-5}}

A stable \(x\) relates to \(y\).

\hypertarget{inference-2-3}{%
\subsection{Inference 2}\label{inference-2-3}}

A flucuating \(x\) relates to \(y\).

\hypertarget{growth-1}{%
\section{Growth}\label{growth-1}}

\hypertarget{inference-1-6}{%
\subsection{Inference 1}\label{inference-1-6}}

There is growth (positive or negative) in a given variable.

\hypertarget{inference-2-4}{%
\subsection{Inference 2}\label{inference-2-4}}

There are inter-individual differences in growth.

\hypertarget{inference-3-2}{%
\subsection{Inference 3}\label{inference-3-2}}

There is a relationship between growth (slope) and level in a given
variable.

\hypertarget{inference-4-1}{%
\subsection{Inference 4}\label{inference-4-1}}

There is a relationship between a stable \(x\) and growth in \(y\).
There are inter-individual characteristics that relate to
inter-individual differences in slope.

\hypertarget{inference-5-1}{%
\subsection{Inference 5}\label{inference-5-1}}

There is a relationship between a fluctuating \(x\) and \(y\) after
partialling the growth in \(y\). Or, there is growth in \(y\) after
partialling the relationship between a fluctuating \(x\) and \(y\).

\hypertarget{inference-6-growth-2.0}{%
\subsection{Inference 6 (growth 2.0)}\label{inference-6-growth-2.0}}

There are correlated slopes among two growth curves.

\hypertarget{change-1}{%
\section{Change}\label{change-1}}

\hypertarget{inference-1-7}{%
\subsection{Inference 1}\label{inference-1-7}}

\(x\) is associated with a change in \(y\): an increase or decrease.

\hypertarget{dynamics-1}{%
\section{Dynamics}\label{dynamics-1}}

\hypertarget{inference-1-8}{%
\subsection{Inference 1}\label{inference-1-8}}

There is autoregression in a variable, a relationship between prior and
future values.

\hypertarget{inference-2-5}{%
\subsection{Inference 2}\label{inference-2-5}}

There are cross-lag effects, where one variable relates to another at a
different point in time.

\hypertarget{inference-3-3}{%
\subsection{Inference 3}\label{inference-3-3}}

There is a reciprocal relationship, or feedback, where one variable
subsequently relates to another, and this second variable then relates
to the first at an even later point in time.

\newpage

\hypertarget{references}{%
\section{References}\label{references}}

\begin{Shaded}
\begin{Highlighting}[]
\KeywordTok{r_refs}\NormalTok{(}\DataTypeTok{file =} \StringTok{"references.bib"}\NormalTok{)}
\end{Highlighting}
\end{Shaded}

\setlength{\parindent}{-0.5in}
\setlength{\leftskip}{0.5in}

\hypertarget{refs}{}
\leavevmode\hypertarget{ref-beal_esm_2015}{}%
Beal, D. J. (2015). ESM 2.0: State of the art and future potential of
experience sampling methods in organizational research. \emph{Annu. Rev.
Organ. Psychol. Organ. Behav.}, \emph{2}(1), 383--407.

\leavevmode\hypertarget{ref-hardy_interrelationships_2018}{}%
Hardy, J. H., Day, E. A., \& Steele, L. M. (2018). Interrelationships
Among Self-Regulated Learning Processes: Toward a Dynamic Process-Based
Model of Self-Regulated Learning. \emph{Journal of Management},
0149206318780440.
doi:\href{https://doi.org/10.1177/0149206318780440}{10.1177/0149206318780440}

\leavevmode\hypertarget{ref-ilgen_computational_2000}{}%
Ilgen, D. R., \& Hulin, C. L. (2000). \emph{Computational modeling of
behavior in organizations: The third scientific discipline.} American
Psychological Association.

\leavevmode\hypertarget{ref-johnson_good_2014}{}%
Johnson, R. E., Lanaj, K., \& Barnes, C. M. (2014). The good and bad of
being fair: Effects of procedural and interpersonal justice behaviors on
regulatory resources. \emph{Journal of Applied Psychology},
\emph{99}(4), 635.

\leavevmode\hypertarget{ref-jones_baby_2016}{}%
Jones, K. P., King, E. B., Gilrane, V. L., McCausland, T. C., Cortina,
J. M., \& Grimm, K. J. (2016). The baby bump: Managing a dynamic stigma
over time. \emph{Journal of Management}, \emph{42}(6), 1530--1556.

\leavevmode\hypertarget{ref-koopman_integrating_2016}{}%
Koopman, J., Lanaj, K., \& Scott, B. A. (2016). Integrating the Bright
and Dark Sides of OCB: A Daily Investigation of the Benefits and Costs
of Helping Others. \emph{Academy of Management Journal}, \emph{59}(2),
414--435.
doi:\href{https://doi.org/10.5465/amj.2014.0262}{10.5465/amj.2014.0262}

\leavevmode\hypertarget{ref-kozlowski_work_2003}{}%
Kozlowski, S. W., \& Bell, B. S. (2003). Work groups and teams in
organizations. \emph{Handbook of Psychology}, 333--375.

\leavevmode\hypertarget{ref-meier_reciprocal_2013}{}%
Meier, L. L., \& Spector, P. E. (2013). Reciprocal effects of work
stressors and counterproductive work behavior: A five-wave longitudinal
study. \emph{Journal of Applied Psychology}, \emph{98}(3), 529.

\leavevmode\hypertarget{ref-pitariu_explaining_2010}{}%
Pitariu, A. H., \& Ployhart, R. E. (2010). Explaining change: Theorizing
and testing dynamic mediated longitudinal relationships. \emph{Journal
of Management}, \emph{36}(2), 405--429.

\leavevmode\hypertarget{ref-ployhart_longitudinal_2010}{}%
Ployhart, R. E., \& Vandenberg, R. J. (2010). Longitudinal research: The
theory, design, and analysis of change. \emph{Journal of Management},
\emph{36}(1), 94--120.

\leavevmode\hypertarget{ref-ritter_understanding_2016}{}%
Ritter, K.-J., Matthews, R. A., Ford, M. T., \& Henderson, A. A. (2016).
Understanding role stressors and job satisfaction over time using
adaptation theory. \emph{Journal of Applied Psychology}, \emph{101}(12),
1655.

\leavevmode\hypertarget{ref-rosen_who_2016}{}%
Rosen, C. C., Koopman, J., Gabriel, A. S., \& Johnson, R. E. (2016). Who
strikes back? A daily investigation of when and why incivility begets
incivility. \emph{Journal of Applied Psychology}, \emph{101}(11), 1620.


\end{document}
