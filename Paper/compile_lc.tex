\documentclass[english,,man]{apa6}
\usepackage{lmodern}
\usepackage{amssymb,amsmath}
\usepackage{ifxetex,ifluatex}
\usepackage{fixltx2e} % provides \textsubscript
\ifnum 0\ifxetex 1\fi\ifluatex 1\fi=0 % if pdftex
  \usepackage[T1]{fontenc}
  \usepackage[utf8]{inputenc}
\else % if luatex or xelatex
  \ifxetex
    \usepackage{mathspec}
  \else
    \usepackage{fontspec}
  \fi
  \defaultfontfeatures{Ligatures=TeX,Scale=MatchLowercase}
\fi
% use upquote if available, for straight quotes in verbatim environments
\IfFileExists{upquote.sty}{\usepackage{upquote}}{}
% use microtype if available
\IfFileExists{microtype.sty}{%
\usepackage{microtype}
\UseMicrotypeSet[protrusion]{basicmath} % disable protrusion for tt fonts
}{}
\usepackage{hyperref}
\hypersetup{unicode=true,
            pdftitle={Inferences With Longitudinal Data},
            pdfauthor={\ldots{}},
            pdfkeywords={\ldots{}.},
            pdfborder={0 0 0},
            breaklinks=true}
\urlstyle{same}  % don't use monospace font for urls
\ifnum 0\ifxetex 1\fi\ifluatex 1\fi=0 % if pdftex
  \usepackage[shorthands=off,main=english]{babel}
\else
  \usepackage{polyglossia}
  \setmainlanguage[]{english}
\fi
\usepackage{graphicx,grffile}
\makeatletter
\def\maxwidth{\ifdim\Gin@nat@width>\linewidth\linewidth\else\Gin@nat@width\fi}
\def\maxheight{\ifdim\Gin@nat@height>\textheight\textheight\else\Gin@nat@height\fi}
\makeatother
% Scale images if necessary, so that they will not overflow the page
% margins by default, and it is still possible to overwrite the defaults
% using explicit options in \includegraphics[width, height, ...]{}
\setkeys{Gin}{width=\maxwidth,height=\maxheight,keepaspectratio}
\IfFileExists{parskip.sty}{%
\usepackage{parskip}
}{% else
\setlength{\parindent}{0pt}
\setlength{\parskip}{6pt plus 2pt minus 1pt}
}
\setlength{\emergencystretch}{3em}  % prevent overfull lines
\providecommand{\tightlist}{%
  \setlength{\itemsep}{0pt}\setlength{\parskip}{0pt}}
\setcounter{secnumdepth}{0}
% Redefines (sub)paragraphs to behave more like sections
\ifx\paragraph\undefined\else
\let\oldparagraph\paragraph
\renewcommand{\paragraph}[1]{\oldparagraph{#1}\mbox{}}
\fi
\ifx\subparagraph\undefined\else
\let\oldsubparagraph\subparagraph
\renewcommand{\subparagraph}[1]{\oldsubparagraph{#1}\mbox{}}
\fi

%%% Use protect on footnotes to avoid problems with footnotes in titles
\let\rmarkdownfootnote\footnote%
\def\footnote{\protect\rmarkdownfootnote}


  \title{Inferences With Longitudinal Data}
    \author{\ldots{}\textsuperscript{1}}
    \date{}
  
\shorttitle{LONGITUDINAL INFERENCES}
\affiliation{
\vspace{0.5cm}
\textsuperscript{1} ...}
\keywords{....\newline\indent Word count: 95}
\usepackage{csquotes}
\usepackage{upgreek}
\captionsetup{font=singlespacing,justification=justified}

\usepackage{longtable}
\usepackage{lscape}
\usepackage{multirow}
\usepackage{tabularx}
\usepackage[flushleft]{threeparttable}
\usepackage{threeparttablex}

\newenvironment{lltable}{\begin{landscape}\begin{center}\begin{ThreePartTable}}{\end{ThreePartTable}\end{center}\end{landscape}}

\makeatletter
\newcommand\LastLTentrywidth{1em}
\newlength\longtablewidth
\setlength{\longtablewidth}{1in}
\newcommand{\getlongtablewidth}{\begingroup \ifcsname LT@\roman{LT@tables}\endcsname \global\longtablewidth=0pt \renewcommand{\LT@entry}[2]{\global\advance\longtablewidth by ##2\relax\gdef\LastLTentrywidth{##2}}\@nameuse{LT@\roman{LT@tables}} \fi \endgroup}


\DeclareDelayedFloatFlavor{ThreePartTable}{table}
\DeclareDelayedFloatFlavor{lltable}{table}
\DeclareDelayedFloatFlavor*{longtable}{table}
\makeatletter
\renewcommand{\efloat@iwrite}[1]{\immediate\expandafter\protected@write\csname efloat@post#1\endcsname{}}
\makeatother
\usepackage{lineno}

\linenumbers

\authornote{\ldots{}.

Correspondence concerning this article should be addressed to \ldots{},
\ldots{}. E-mail: \ldots{}}

\abstract{
Begin here\ldots{}


}

\usepackage{amsthm}
\newtheorem{theorem}{Theorem}[section]
\newtheorem{lemma}{Lemma}[section]
\theoremstyle{definition}
\newtheorem{definition}{Definition}[section]
\newtheorem{corollary}{Corollary}[section]
\newtheorem{proposition}{Proposition}[section]
\theoremstyle{definition}
\newtheorem{example}{Example}[section]
\theoremstyle{definition}
\newtheorem{exercise}{Exercise}[section]
\theoremstyle{remark}
\newtheorem*{remark}{Remark}
\newtheorem*{solution}{Solution}
\begin{document}
\maketitle

Organizational phenomena unfold over time. They are processes that
develop, change, and evolve (Pitariu \& Ployhart, 2010) that create a
sequence of events within a person's stream of experience (Beal, 2015).
Moreover, organizations are systems with many connected parts, and
systems are inherently dynamic. Studying these systems and processes,
therefore, requires that we attend not to static snapshots of behavior
(Ilgen \& Hulin, 2000), but to variables and relationships as they move
through time; doing so puts us in a better position to capture the
sequence, understand it, and can lead to new and interesting insights
(Kozlowski \& Bell, 2003).

\textbf{Option A\ldots{}It's hard for newcomers}

Although our field is increasingly interested in exploring patterns in
longitudinal data, process-oriented methods are still relatively new to
our field and newcomers without much longitudinal modeling training may
be unfamiliar with the variety of questions they can ask. Consider a few
recent longitudinal studies that all pose different questions. Jones et
al. (2016) ask if the trajectories among certain variables increase or
decrease over time. Johnson, Lanaj, and Barnes (2014) study how changes
in one variable relate to changes in another across time. Hardy, Day,
and Steele (2018) inquire about dynamic relationships, where prior
values on one variable predict subsequent values on another, and this
second variable then goes back to predict the first at a later point in
time. Finally, Meier and Spector (2013) examine how effect sizes change
when they vary the time lag between their independent and dependent
variable.

There are then complex statistical models that researchers evoke to
examine their questions. Meier and Spector (2013) present a sequence of
path models that test increasingly longer time lags. Hardy et al. (2018)
and Jones et al. (2016) employ bivariate cross-lagged latent growth
curves, an approach similar to the latent change model used by Ritter,
Matthews, Ford, and Henderson (2016). We also find complex hierarchical
linear models in many event-sampling studies (e.g., Koopman, Lanaj, \&
Scott, 2016; Rosen, Koopman, Gabriel, \& Johnson, 2016). Again,
researchers without much longitudinal modeling training may not know
when to apply each model -- or which model is appropriate for a given
question.

Finally, the spine of an investigation is to interpret the model and
make an inference regarding the original question. Jones et al. (2016)
infer negative slopes for concealing behaviors and positive slopes for
revealing behaviors. Johnson et al. (2014) state that justice behaviors
fluctuate day to day and predict changes in depletion. Hardy et al.
(2018) find support for dynamic relationships between self-efficacy,
metacognition, and exploratory behaviors. Finally, Meier and Spector
(2013) suggest that the effects of work stressors on counterproductive
work behaviors are not substantially different across different time
lags.

In this paper we discuss the common inferences that researchers in our
field make when they apply a model to longitudinal data. As should be
clear to anyone reading our literature, there is great excitement for
the utility of longitudinal studies; they can pose interesting questions
and discover patterns that would otherwise be impossible to capture in a
static investigation. We bring attention to the span of questions
available so that researchers can fully appreciate and take advantage of
their data. Although the inferences concern trajectories or
relationships over time, their small differences have large implications
for what we take away from them -- what we ultimately conclude.
Moreover, there are many inferences, many models, and different models
can be used to understand or explore the same inference. In this paper,
we provide readers with potential models for each inference so that they
can be sure that the model they evoke is appropriate for the research
question that they are interested in. In summary, this paper exposes
researchers to the span of inferences they may investigate when they
collect longitudinal data, links those inferences to models, and parses
some of the modeling literature that may be difficult to consume for
researchers with only graduate level training in statistics.

\textbf{Option B\ldots{}Just highlighting the literature}

There are many interesting questions researchers can explore with
longitudinal data. Consider a few recent longitudinal studies that all
pose different questions. Jones et al. (2016) ask if the trajectories
among certain variables increase or decrease over time. Johnson et al.
(2014) study how changes in one variable relate to changes in another
across time. Hardy et al. (2018) inquire about dynamic relationships,
where prior values on one variable predict subsequent values on another,
and this second variable then goes back to predict the first at a later
point in time. Finally, Meier and Spector (2013) examine how effect
sizes change when they vary the time lag between their independent and
dependent variable.

There are then complex statistical models that researchers evoke to
examine their questions. Meier and Spector (2013) present a sequence of
path models that test increasingly longer time lags. Hardy et al. (2018)
and Jones et al. (2016) employ bivariate cross-lagged latent growth
curves, an approach similar to the latent change model used by Ritter et
al. (2016). We also find complex hierarchical linear models in many
event-sampling studies (e.g., Koopman et al., 2016; Rosen et al., 2016).

Finally, the spine of an investigation is to interpret the model and
make an inference regarding the original question. Jones et al. (2016)
infer negative slopes for concealing behaviors and positive slopes for
revealing behaviors. Johnson et al. (2014) state that justice behaviors
fluctuate day to day and predict changes in depletion. Hardy et al.
(2018) find support for dynamic relationships between self-efficacy,
metacognition, and exploratory behaviors. Finally, Meier and Spector
(2013) suggest that the effects of work stressors on counterproductive
work behaviors are not substantially different across different time
lags.

In this paper we discuss the common inferences that researchers in our
field make when they apply a model to longitudinal data. As should be
clear to anyone reading our literature, there is great excitement for
the utility of longitudinal studies; they can pose interesting questions
and discover patterns that would otherwise be impossible to capture in a
static investigation. We bring attention to the span of questions
available so that researchers can fully appreciate and take advantage of
their data. Although the inferences concern trajectories or
relationships over time, their small differences have large implications
for what we take away from them -- what we ultimately conclude.
Moreover, there are many inferences, many models, and different models
can be used to understand or explore the same inference. In this paper,
we provide readers with potential models for each inference so that they
can be sure that the model they evoke is appropriate for the research
question that they are interested in. In summary, this paper exposes
researchers to the span of inferences they may investigate when they
collect longitudinal data, links those inferences to models, and parses
some of the modeling literature that may be difficult to consume for
researchers with only graduate level training in statistics.

\hypertarget{longitudinal-definitions}{%
\section{Longitudinal Definitions}\label{longitudinal-definitions}}

This paper is exclusively devoted to the inferences we make with
repeated observations, so we begin by identifying a few labels and
definitions. Authors typically identify a \enquote{longitudinal} study
by contrasting either a) research designs or b) data structures.
Longitudinal \emph{research} is different from cross-sectional research
because longitudinal designs entail three or more repeated observations
(Ployhart \& Vandenberg, 2010). We therefore emphasize differences on
the number of observations when we distinguish longitudinal from other
types of research. Longitudinal \emph{data} are repeated observations on
several units (i.e., \(N\) or \(i\) \textgreater{} 1), whereas panel
data are observations of one unit over time -- a distinction that
focuses on the amount of people in our study (given repeated measures).
Most organizational studies collect data on more than one unit,
therefore our discussion below focuses on longitudinal research with
longitudinal data, or designs with \(N\) \textgreater{} 1, \(t\)
\textgreater{}= 3, and the same construct(s) measured on each \(i\) at
each \(t\).

\hypertarget{framework}{%
\section{Framework}\label{framework}}

We use three inference categories to partition our discussion, including
level, trend, and dynamics. Each of these are broad categories, and they
will all have their own sub-inferences and models. Our writing style
will be the same throughout each section, where we first discuss the
category itself and then sequentially walk through the inferences.
During that sequence, we will pose questions to orient the reader as to
what the inference captures, unpack graphs and figures, and then close
with a table that provides example hypotheses that align with each
inference. The figures we use in the level and trend sections are graphs
that show what the inferences look like in data -- we feel that graphing
the inferences with respect to data is more informative than your usual
box and arrow diagram. There is a caveat, however, that we want to make
sure everyone is aware of. Data are always messy. It is rare to find
data where the inferences expose themselves simply by plotting --
although it is certainly possible. We are using these \enquote{data
plots} to clearly convey what the inferences mean, but please be aware
that field data will always be messy. When we discuss dynamics section
we then use box and arrow diagrams because they better convey the ideas
in that section.

Finally, we end each inference section by pointing researchers to
respective statistical models. Although we direct researchers to models,
our paper is not about statistical modeling only -- it is about
inferences -- and researchers therefore need to be sure that they
appreciate all of the nuance before applying a recommended model. There
are many complex statistical issues that arise with longitudinal
modeling -- like stationarity -- and the models differ in how they
handle these issues, the assumptions they make, and the data format they
require. There are plenty of great references on each model, what we are
doing here is guiding researchers to those references based on the
underlying inferences that interest them.

\hypertarget{level}{%
\section{Level}\label{level}}

Is employee emotional exhaustion, on average, high across the study? Is
trainee skill low at the beginning of a training session? What value are
newcomer perceptions of unit climate at the end of a two-week
socialization process? These are questions about level, or the specific
value of a variable.

Levels either describe the variable at one moment or averaged across a
span of time. That is, if we put a variable on the \(y\) axis and plot
its values against time on the \(x\) axis, we can explore the value that
it takes at time \(t\), or the value that it takes on average across any
span of \(t\).

Figure \ref{level} demonstrates this idea graphically. A variable is
plotted across time for a single person (i.e., unit), and the color
labels indicate levels -- the red and green describe the variable at a
specific moment while the purple, average level, describes it across a
window. The starting level is the value of the variable at the first
time point, the ending level is its value at the last time point, and
the average level is its average level across time.

\begin{center}

------------

Insert Figure \ref{level} about here

------------

\end{center}

\noindent Our first level inference, therefore, concerns the value of a
variable at a specific time or averaged across a window of time.

\begin{quote}
\begin{quote}
\textbf{Inference 1:} What is the level of \(x\) at time \(t\), or
across a span of \(t\)?
\end{quote}
\end{quote}

When we retain one variable but add multiple units -- people or
organizations, for example -- then we can examine variability in level.
Does everyone have high affect across time? Is there variability in the
level of skill among trainees at the beginning of a training session?

We demonstrate this idea in figure \ref{level_var}, where we now plot
three units (people) across time. Each individual has a similar
trajectory, but their ending levels of \(y\) are different. Said
formally that is, \enquote{there is variability across units in level at
the last time point.}

\begin{center}

------------

Insert Figure \ref{level_var} about here

------------

\end{center}

\noindent The second level inference, therefore, is about level
variability across units.

\begin{quote}
\begin{quote}
\textbf{Inference 2:} Across a span of \(t\) or at a specific \(t\)
there is variability in the level of \(x\).
\end{quote}
\end{quote}

Inferences one and two concern a single variable, but they can of course
be iterated across any or all observed variables in the study (remember
that variables are different than units). For example, we might discover
that affect and performance have high average levels across time, but
that affect has greater level variability across units. Or we might
learn that affect has a low initial level whereas performance is
initially high. What we are doing here is making descriptive comparisons
between the level of one variable and the level of another. We can also
produce a quantitative statement about the extent to which levels are
related.

Correlating levels provides us with that quantitative statement. A large
positive correlation between the initial levels of affect and
performance would mean that people with greater initial levels of affect
also tend to have greater intial performance, and people with lower
initial affect also tend to have lower initial performance.

Figure \ref{level_correlate} demonstrates a correlation of starting
levels. In Panel A we plot affect and performance trajectories for three
individuals across time, where the black solid line indicates affect and
the dashed line indicates performance. We indicate starting levels for
both variables in Panel A by placing colored circles on the graph for
each individual. For example, we indicate the starting levels of affect
and performance for person one with red circles and the starting levels
for person two with green circles.

Panel B of figure \ref{level_correlate} maps those starting levels onto
a new plot that leads to our inference. On the \(x\)-axis is initial
level of affect, where high values indicate a high starting level of
affect, and on the \(y\)-axis is initial level of performance, where
high values indicate a high starting level of performance. The red
circle for person one is on the bottom right because that individual has
a high initial level of affect and a low initial level of performance.
Person two (the green circle) is on the top left because that individual
has a high initial level of performance and a low initial level of
affect, and person three is in the middle because they have roughly the
same starting levels of affect and performance. The dots slope downward
in Panel B, which tells us that there is a negative relationship between
initial level of affect and initial level of performance.

Overall, figure \ref{level_correlate} suggests that the starting levels
of affect and performance are correlated. Panel A shows the actual
starting levels, and Panel B shows that there is a strong negative
correlation between initial affect and initial performance. This
negative relationship means that we expect people with low initial
affect to have high initial performance, whereas we expect people with
high initial affect to have low initial performance.

\begin{quote}
\begin{quote}
\textbf{Inference 3:} There is a correlation between the level of \(x\)
and the level of \(y\) at \(t\).
\end{quote}
\end{quote}

In our final level inference we correlate values across time rather than
correlating values from a single moment or a single averaged moment. For
example, we might ask if affect is related to performance across time;
i.e., when affect is high is performance also high, and when affect is
low is performance also low?

This inference sounds similar to the one just presented, but their
difference is important. With inference three we ask about affect and
performance at \(t\) or at an averaged window of \(t\) -- we examine,
for example, how ending performance relates to ending affect, or how
affect averaged across time relates to performance averaged across time.
Here, we retain all of the information and examine the relationship
between affect and performance across all \(t\).

Figure \ref{level_relation} shows this inference graphically. In Panel A
we plot affect and performance trajectories across time, where the solid
line indicates affect and the dashed line indicates performance -- this
time we only focus on a single individual or unit. The colored squares
represent levels at different points in time. The green squares
highlight low values of both variables, the blue high values, and the
red middle values.

Panel B shows how those respective values map onto a scatterplot of
affect and performance -- which again will lead to the inference. The
blue values indicate that high values of affect tend to co-occur with
high values of performance (shown respectively by the blue squares in
Panel A). The red values indicate that middle values of affect tend to
co-occur with middle values of performance. The green values, finally,
indicate that low values of affect tend to co-occur with low values of
performance. Across time, affect and performance covary.

\begin{center}

------------

Insert Figure \ref{level_relation} about here

------------

\end{center}

\begin{quote}
\begin{quote}
\textbf{Inference 4:} There is a relationship between \(x\) and \(y\)
across time.
\end{quote}
\end{quote}

\hypertarget{level-inference-table}{%
\subsection{Level Inference Table}\label{level-inference-table}}

The inference table below provides examples of each level inference.
Inference one is about level itself -- a single value that describes the
variable at one time or averaged across time. Inference two is about
variability across units in level. Inference three takes the level in
one variable and asks whether it tends to co-occur with the level in
another. Think of inference three as creating a latent level variable at
a single moment and asking how it relates to another latent variable
from a single moment. Inference four, finally, is about the relationship
between raw values across time.

\begin{center}

------------

Insert Table \ref{level_table} about here

------------

\end{center}

\hypertarget{models}{%
\subsection{Models}\label{models}}

Level is called intercept in the statistical modeling literature.
Typically the mean estimate tells you about the level, and the variance
estimate tells you about the variability across units. Intercept only
models in HLM or SEM. Time-varying or invariant covariates analyses.
Point to references.

\begin{table}

\caption{\label{tab:unnamed-chunk-6}\label{level_table}Examples of level inferences.}
\centering
\begin{tabular}[t]{>{\raggedright\arraybackslash}p{5em}>{\raggedright\arraybackslash}p{30em}}
\toprule
Inference & Examples\\
\midrule
1 & Burnout is high at the last time point.\newline Performance is low, on average, across time.\\
\hline
2 & Average affect across time differs across people (units).\newline There is variability in the initial level of turnover across organizations.\\
\hline
3 & People with greater initial health status also have greater initial happiness.\newline People with high performance on average across time have lower anxiety on average across time.\\
\hline
4 & Affect relates to performance across time.\newline Helping behaviors predict depletion across time.\\
\bottomrule
\end{tabular}
\end{table}

\begin{figure}
\centering
\includegraphics{figures/unnamed-chunk-7-1.pdf}
\caption{\label{fig:unnamed-chunk-7}Level examples\label{level}}
\end{figure}

\begin{figure}
\centering
\includegraphics{figures/unnamed-chunk-8-1.pdf}
\caption{\label{fig:unnamed-chunk-8}Trajectories with variability in ending
level across units\label{level_var}}
\end{figure}

\begin{figure}
\centering
\includegraphics{figures/unnamed-chunk-9-1.pdf}
\caption{\label{fig:unnamed-chunk-9}Correlating starting levels, or relating
initial affect to initial performance\label{level_correlate}}
\end{figure}

\begin{figure}
\centering
\includegraphics{figures/unnamed-chunk-10-1.pdf}
\caption{\label{fig:unnamed-chunk-10}Relating affect to performance
levels\label{level_relation}}
\end{figure}

\hypertarget{trend}{%
\section{Trend}\label{trend}}

Does affect go up or down across time, or is it relatively stable? Does
trainee skill increase over the training session? These are questions
about trend, and these first two are specifically about linear trend. It
is also possible to explore how variables bend or curve across time. Do
newcomer perceptions of climate increase and then plateau over time?
Does the response time of a medical team decrease with each successive
case but then remain stable once the team can no longer improve their
coordination? These latter questions concern curvilinear trajectories.

Trend has to do with the global shape of the trajectory across time. If
we put a variable on the \(y\)-axis and plot its values against time on
the \(x\)-axis, do the values tend to go up or down over time? It can be
thought of as the coarse-grained direction of a trajectory.

Figure \ref{trend} demonstrates trend, where the red line shows
negative, decreasing trend, the blue line shows positive, increasing
trend, and the green line shows a curvilinear trajectory. Keep in mind
that curvilinear and linear trajectories are both \emph{linear in
parameters} and should not be confused with non-linear systems.

\begin{center}

------------

Insert Figure \ref{trend} about here

------------

\end{center}

Our first trend inference, therefore, concerns the shape of the
trajectory.

\begin{quote}
\begin{quote}
\textbf{Inference 1:} There is positive/negative/curvilinear trend in a
variable across time.
\end{quote}
\end{quote}

As with the level inferences, when we add more units we can examine
trend variability. Do all trainees develop greater skill across time? Is
there variability in the trend of helping behaviors, or
counterproductive work behaviors over time?

Figure \ref{trend_var} shows differences in trend variability. In the
first facet all units (people) show the same positive trend, whereas
everyone in the second facet shows different trend: person one's data
appear to increase over time, person two's data decrease over time, and
person three's data remain flat. With greater variability there is less
consistency in trend across units.

\begin{center}

------------

Insert Figure \ref{trend_var} about here

------------

\end{center}

\begin{quote}
\begin{quote}
\textbf{Inference 2:} There is variability in the trend of a variable
across time; trend differs across units.
\end{quote}
\end{quote}

Inferences one and two are about one variable, but they can also be
iterated across all observed variables. For example, we might discover
that affect and performance trends both decrease, but there is greater
variability across units in the affect trend. Or we might learn that
affect has a negative trend while performance has a positive trend.

Correlating these trends is the next inference. Correlating indicates
co-occuring patterns, but this time we are focused on trends rather than
levels. A large positive correlation between affect and performance
trends indicates that people with a positive affect trend (usually) have
a positive performance trend and people with a negative affect trend
(usually) have a negative performance trend.

Figure \ref{trend_correlation} shows the inuition behind the inference
with a set of graphs. In Panel A we plot affect and performance across
time for three individuals. Affect goes up while performance goes down
for person one, this pattern is reversed for person two, and person
three reports trendless affect and performance (i.e., zero trend), but
both variables fluctuate across time for this individual. We use
different colors to label the trends for each person. The affect and
performance trends for person one are labeled with red lines, whereas
person two has green lines and person three has blue lines.

Panel B then maps those pairings onto a figure that shows the
relationship between the affect and performance trend. For example,
person one has a positive affect trend and a negative performance trend,
so their value in Panel B goes on the bottom right, whereas person two
has the opposite pattern and therefore is placed on the top left (where
the performance trend is positive and the affect trend is negative).
Producing this bottom scatter plot tells us that the relationship
between the affect and performance trend is negative. That is, people
with a positive affect trend usually have a negative performance trend,
people with a negative affect trend are more likely to have a positive
performance trend, and people with no affect trend usually have no
performance trend.

\begin{center}

------------

Insert Figure \ref{trend_correlation} about here

------------

\end{center}

\begin{quote}
\begin{quote}
\textbf{Inference 3:} There are correlated trends.
\end{quote}
\end{quote}

The final trend inference is about identifying covariates or predictors
of trend. Does gender predict depletion trends? Does the trend in unit
climate covary with between unit differences in leader quality? Notice
the difference between this inference and inference three. Inference
three asks how one trend is related to another, whereas this inference
asks how one trend relates to a covariate.

Figure \ref{trend_covariate} demonstrates the inference in a plot. We
plot affect across time for six employees, and these employees differ by
job type. The first three individuals work in research and development,
whereas the final three individuals work in sales. Affect trajectories
tend to decrease over time for employees in research and development,
whereas affect trajectories tend to increase for those in sales. An
individual's job type, then, gives us a clue to their likely affect
trend -- said formally, job type covaries with affect trends, such that
we expect individuals in sales to have positive affect trends and
individuals in research and development to have negative affect trends.
The expected trends are plotted as the thick blue lines.

\begin{quote}
\begin{quote}
\textbf{Inference 4:} There are correlates of trend.
\end{quote}
\end{quote}

\hypertarget{trend-inference-table}{%
\subsection{Trend Inference Table}\label{trend-inference-table}}

The inference table below provides examples of each trend inference.
Inference one is about the general direction or shape of a trajectory
across time. Inference two is about variability in that shape across
units. Inference three takes the trend in one variable and asks whether
it co-occurs with trend in another. Inference four, finally, is about
the relationship between trend in one variable and the raw values of one
or more correlates.

\begin{center}

------------

Insert Table \ref{trend_table} about here

------------

\end{center}

We want to close this section with a note on phrasing. The inferences we
explored in this section have to do with correlating trends, where a
statement like \enquote{affect and performance trends covary, such that
people with a negative affect trend have a positive performance trend}
is appropriate. There is a less precise phrase that is easy to fall into
-- and we have seen it used in our literature. Sometimes, researchers
will correlate trends and then state, \enquote{when affect decreases
performance goes up.} We urge researchers to avoid this second statement
because it is not clear if it refers to a static relationship about
trends or a dynamic statement about how trajectories move across time.
That is, the phrase \enquote{when affect decreases performance goes up}
could refer to correlated trends, but it could also mean something like,
\enquote{when affect decreases performance immediately or subsequently
goes up.} This second statement is far different and it should not be
used when an analysis only correlates trends or evokes predictors of
trend. Again, we urge researchers to phrase their inferences as we have
shown here.

\hypertarget{models-1}{%
\subsection{Models}\label{models-1}}

Trends are called slope estimates in the statistical modeling
literature. They are also referred to as growth. Mean estimates of
slopes, or trends, or growth will tell you about trend, whereas the
variance estimates will tell you about variability across units. Growth
curves in SEM or HLM. Bivariate growth curves.

\begin{table}

\caption{\label{tab:unnamed-chunk-11}\label{trend_table}Examples of trend inferences.}
\centering
\begin{tabular}[t]{>{\raggedright\arraybackslash}p{5em}>{\raggedright\arraybackslash}p{30em}}
\toprule
Inference & Examples\\
\midrule
1 & Burnout decreases over time.\newline Performance increases over time.\\
\hline
2 & Affect trends differ across people (units).\newline There is variability in turnover trends across organizations.\\
\hline
3 & People with a positive health status trend have a positive happiness trend.\newline People with a positive performance trend have a negative anxiety trend.\\
\hline
4 & Gender correlates with depletion trends.\newline Unit climate covaries with unit performance trends.\\
\bottomrule
\end{tabular}
\end{table}

\begin{figure}
\centering
\includegraphics{figures/unnamed-chunk-12-1.pdf}
\caption{\label{fig:unnamed-chunk-12}Trend across time\label{trend}}
\end{figure}

\begin{figure}
\centering
\includegraphics{figures/unnamed-chunk-13-1.pdf}
\caption{\label{fig:unnamed-chunk-13}Differences in trend variability across
units\label{trend_var}}
\end{figure}

\begin{figure}
\centering
\includegraphics{figures/unnamed-chunk-14-1.pdf}
\caption{\label{fig:unnamed-chunk-14}Correlating slopes, or relating the
affect to performance trend\label{trend_correlation}}
\end{figure}

\begin{figure}
\centering
\includegraphics{figures/unnamed-chunk-15-1.pdf}
\caption{\label{fig:unnamed-chunk-15}Job type as a covariate of affect
trend\label{trend_covariate}}
\end{figure}

\hypertarget{dynamics}{%
\section{Dynamics}\label{dynamics}}

Dynamics refers to a specific branch of mathematics, but the term is
used in different ways throughout our literature. It is used informally
to mean \enquote{change}, \enquote{fluctuating,} \enquote{volatile,}
\enquote{longitudinal,} or \enquote{over time} (among others), whereas
formal definitions in our literature are presented within certain
contexts. Wang (2016) defines a dynamic \emph{model} as a
\enquote{representation of a system that evolves over time. In
particular it describes how the system evolves from a given state at
time \emph{t} to another state at time \emph{t + 1} as governed by the
transition rules and potential external inputs} (p.~242). Vancouver,
Wang, and Li (2018) state that dynamic \emph{variables} \enquote{behave
as if they have memory; that is, their value at any one time depends
somewhat on their previous value} (p.~604). Finally, Monge (1990)
suggests that in dynamic \emph{analyses}, \enquote{it is essential to
know how variables depend upon their own past history} (p.~409).

The crucial notion to take from dynamics, then, is memory. When the past
matters, and future states are constrained by where they were at prior
points in time, dynamics are at play. In this section, we unpack a
variety of inferences that are couched in this idea.

Does performance relate to itself over time? Do current helping
behaviors depend on prior helping behaviors? Does unit climate
demonstrate self-similarity across time? Does income now predict income
in the future? These are questions about the relationship of a single
variable with itself over time -- does it predict itself at each
subsequent moment? Is it constrained by where it was in the past?

Panel A of figure \ref{dynamics_figure} shows the concept with a box and
arrow diagram. \(x\) predicts itself across every moment -- it has
self-similarity and its value now is constrained by where it was a
moment ago. In our diagram we show that \(x\) at time \(t\) is related
to \(x\) at time \(t + 1\). In other words, we posit that \(x\) shows a
lag-one relationship, where \(x\) is related to its future value one
time step away. Modelers and theorists are of course free to suggest any
lag amount that they believe captures the actual relationship.

\begin{quote}
\begin{quote}
\textbf{Inference 1:} There is self-similarity in \(x\); \(x\) relates
to itself across time.
\end{quote}
\end{quote}

Inference one was about a single variable, and in the level and trend
inference sections we saw that when we moved to multiple variables we
started asking how variables relate to one another at \(t\), or at an
average window of \(t\), or across \(t\). With dynamics, where memory is
a fundamental concept, we instead ask how variables relate to one
another at different lags. Does affect predict subsequent performance?
Do prior counterproductive work behaviors relate to current incivility?
When goal discrepancy is large is effort at the subsequent time point
high? When prior depletion is low, is current emotional exhaustion high?

We can capture this second inference by relating current values on one
variable to future values on another. Equivalently, we can relate prior
values on one variable to current values on another. Panel B of figure
\ref{dynamics_figure} shows this second dynamics inference. \(x\) still
shows self-similarity across time, but it now predicts \(y\) at the
subsequent moment. We are positing a lag-one relationship between \(x\)
and \(y\). Said formally, we believe that \(x_t\) is related to
\(y_{t+1}\) (or equivalently, \(x_{t-1}\) is related to \(y_{t}\)).
Relating current to future (or prior to current) values from one
variable to another is called a \enquote{cross lag} relationship.

\begin{quote}
\begin{quote}
\textbf{Inference 2:} There is a cross-lag relationship, where one
variable relates to another at a different point in time.
\end{quote}
\end{quote}

Inference two tells us whether the patterns in one variable co-occur
with the patterns in another at a subsequent time point. Across time,
when affect is low is subsequent performance also low? A related
question is as follows: across time, when affect is low does performance
increase or decrease? This second question is about change. How does one
variable relate to the change in another?

When goal discrepancy is large does effort increase or decrease? When
unit climate is low do perceptions of the leader change? When
performance is high does self efficacy go up or down?

All of these questions are about change, but notice that change can be
construed across different lags. Change from what? Baseline? The prior
time point? The last three time points? Typically change is construed
with respect to the last time point. When affect is low, does
performance from the last to the current time point increase or
decrease? How does effort change from the prior to the current time
point when goal discrepancy is high?

Panel C of figure \ref{dynamics_figure} demonstrates this idea. We are
positing the same self-similarity in \(x\) and the same cross-lag
relationship that we saw before, but now \(y\) also has self-similarity
across time. The cross-lag relationship, therefore, is now capturing how
\(y\) has changed from the last point in time.

It is typical to think of change from the prior to the current time
point, but researchers are free to move it as they please. Here are the
two final inferences that capture change in different locations.

\begin{quote}
\begin{quote}
\textbf{Inference 3:} There is a change relationship, where one variable
relates to the change in another.
\end{quote}
\end{quote}

\begin{quote}
\begin{quote}
\textbf{Inference 4:} There is a cross-lag relationship of change, where
one variable relates to the change of another at a different point in
time.
\end{quote}
\end{quote}

\hypertarget{dynamics-inference-table}{%
\subsection{Dynamics Inference Table}\label{dynamics-inference-table}}

Again, we provide an inference table below -- this time with respect to
dynamic inferences. Inference one is about autoregression, or memory in
a single variable. Inference two asks how a variable at one time
co-occurs with another at a different time. Inferences three and four
focus on change: when one variable is high or low, does it relate to the
change (an increase or decrease) in the values of another variable?

\begin{center}

------------

Insert Table \ref{dynamics_table} about here

------------

\end{center}

\hypertarget{extensions}{%
\subsection{Extensions}\label{extensions}}

We described a simple set of inferences above, but the ideas generalize
to more complex dynamics as well. Often researchers are interested in
reciprocal relationships, where \(x\) influences subsequent \(y\), which
then goes back to influence \(x\) at the next time point. Said formally,
\(x_t\) influences \(y_{t+1}\), which then influences \(x_{t+2}\). Said
informally, current performance influences subsequent self-efficacy,
which then influences performance on the next trial. These inferences
are no different than what we saw above -- they are cross-lag
predictions -- all we did here was add more of them. Panel D of figure
\ref{dynamics_figure} shows reciprocal dynamics, where both \(x\) and
\(y\) show self-similarity and cross-lag relationships with one another.

Moreover, the dynamic inferences shown here generalize to systems of
variables, where a researcher posits self-similarity and cross-lag
predictions across many variables. There could be reciprocal dynamics
between a set of variables like performance, self-efficacy, and affect.
There could be a sequence of influence where initial dyadic exchanges
influence subsequent team perceptions, which then influences later
performance, and performance changes the structure of task which
ultimately initiates new dyadic exchanges. Once a researcher grasps the
foundational ideas presented here he or she is free to explore any
number of complex relationships.

Also notice that complex dynamics subsume the concept of mediation. It
is of course an important idea, but when we focus on systems of
variables and reciprocal dynamics we place our emphasis elsewhere. If
readers are interested in mediation we urge them to read one of the many
great papers on it (cites).

\hypertarget{models-2}{%
\subsection{Models}\label{models-2}}

Autoregression is the statistical word for the estimate of
self-similarity in a variable, the relationship between a variable now
and its future value. Our literature has a history with difference
scores and partialling. We debated difference scores so we have
converged to partialling models. Typically we create a model with prior
values of the response variable as a predictor.

\begin{table}

\caption{\label{tab:unnamed-chunk-16}\label{dynamics_table}Examples of dynamic inferences.}
\centering
\begin{tabular}[t]{>{\raggedright\arraybackslash}p{5em}>{\raggedright\arraybackslash}p{30em}}
\toprule
Inference & Examples\\
\midrule
1 & Burnout demonstrates self-similarity across time.\newline Performance relates to subsequent performance.\\
\hline
2 & Affect predicts subsequent counterproductive work behaviors.\newline Turnover relates to subsequent firm performance.\\
\hline
3 & Positive health status relates to change in happiness.\newline Anxiety relates to change in performance.\\
\hline
4 & Affect relates to subsequent change in performance.\newline Helping behaviors predict subsequent depletion changes.\\
\bottomrule
\end{tabular}
\end{table}

\begin{figure}

{\centering \includegraphics[width=3.75in]{figures/dynamics/dall} 

}

\caption{Univariate and bivariate dynamics adapted from DeShon (2012). Panel A shows self-similarity or autoregression in $X$ across time. Panel B shows $X$ predicting subsequent $Y$. Panel C shows $X$ predicting subsequent change in $Y$. Panel D shows reciprocal dynamics between $X$ and $Y$.\label{dynamics_figure}}\label{fig:unnamed-chunk-17}
\end{figure}

\hypertarget{discussion}{%
\section{Discussion}\label{discussion}}

Summary paragraph. We talked about these things.

Other possible discussion pieces. 1) Keep in mind you might see weird
stuff in the literature. X at time 1 relates to Z at time 2, which
relates to Y at time 3, but none are measured repeatedly across time.
This is no good. We opened with \enquote{we couch ourselves by only
discussing studies where constructs were measured on each \(i\) at each
\(t\).} Sometimes this doesn't happen\ldots{} 2) Econometrics modeling
levels vs.~modeling differences.

A section about our opinions of static versus dynamic research. We don't
want to get into the difference between explaining a mechanism vs
describing an observed \enquote{longitudinal} pattern, and we don't want
to say that static research is useless\ldots{}but can we close with some
of our opinions? Some of the ways we hope researchers will go?

\newpage

\hypertarget{references}{%
\section{References}\label{references}}

\setlength{\parindent}{-0.5in}
\setlength{\leftskip}{0.5in}

\hypertarget{refs}{}
\leavevmode\hypertarget{ref-beal_esm_2015}{}%
Beal, D. J. (2015). ESM 2.0: State of the art and future potential of
experience sampling methods in organizational research. \emph{Annu. Rev.
Organ. Psychol. Organ. Behav.}, \emph{2}(1), 383--407.

\leavevmode\hypertarget{ref-hardy_interrelationships_2018}{}%
Hardy, J. H., Day, E. A., \& Steele, L. M. (2018). Interrelationships
Among Self-Regulated Learning Processes: Toward a Dynamic Process-Based
Model of Self-Regulated Learning. \emph{Journal of Management},
0149206318780440.
doi:\href{https://doi.org/10.1177/0149206318780440}{10.1177/0149206318780440}

\leavevmode\hypertarget{ref-ilgen_computational_2000}{}%
Ilgen, D. R., \& Hulin, C. L. (2000). \emph{Computational modeling of
behavior in organizations: The third scientific discipline.} American
Psychological Association.

\leavevmode\hypertarget{ref-johnson_good_2014}{}%
Johnson, R. E., Lanaj, K., \& Barnes, C. M. (2014). The good and bad of
being fair: Effects of procedural and interpersonal justice behaviors on
regulatory resources. \emph{Journal of Applied Psychology},
\emph{99}(4), 635.

\leavevmode\hypertarget{ref-jones_baby_2016}{}%
Jones, K. P., King, E. B., Gilrane, V. L., McCausland, T. C., Cortina,
J. M., \& Grimm, K. J. (2016). The baby bump: Managing a dynamic stigma
over time. \emph{Journal of Management}, \emph{42}(6), 1530--1556.

\leavevmode\hypertarget{ref-koopman_integrating_2016}{}%
Koopman, J., Lanaj, K., \& Scott, B. A. (2016). Integrating the Bright
and Dark Sides of OCB: A Daily Investigation of the Benefits and Costs
of Helping Others. \emph{Academy of Management Journal}, \emph{59}(2),
414--435.
doi:\href{https://doi.org/10.5465/amj.2014.0262}{10.5465/amj.2014.0262}

\leavevmode\hypertarget{ref-kozlowski_work_2003}{}%
Kozlowski, S. W., \& Bell, B. S. (2003). Work groups and teams in
organizations. \emph{Handbook of Psychology}, 333--375.

\leavevmode\hypertarget{ref-meier_reciprocal_2013}{}%
Meier, L. L., \& Spector, P. E. (2013). Reciprocal effects of work
stressors and counterproductive work behavior: A five-wave longitudinal
study. \emph{Journal of Applied Psychology}, \emph{98}(3), 529.

\leavevmode\hypertarget{ref-monge_theoretical_1990}{}%
Monge, P. R. (1990). Theoretical and analytical issues in studying
organizational processes. \emph{Organization Science}, \emph{1}(4),
406--430.

\leavevmode\hypertarget{ref-pitariu_explaining_2010}{}%
Pitariu, A. H., \& Ployhart, R. E. (2010). Explaining change: Theorizing
and testing dynamic mediated longitudinal relationships. \emph{Journal
of Management}, \emph{36}(2), 405--429.

\leavevmode\hypertarget{ref-ployhart_longitudinal_2010}{}%
Ployhart, R. E., \& Vandenberg, R. J. (2010). Longitudinal research: The
theory, design, and analysis of change. \emph{Journal of Management},
\emph{36}(1), 94--120.

\leavevmode\hypertarget{ref-ritter_understanding_2016}{}%
Ritter, K.-J., Matthews, R. A., Ford, M. T., \& Henderson, A. A. (2016).
Understanding role stressors and job satisfaction over time using
adaptation theory. \emph{Journal of Applied Psychology}, \emph{101}(12),
1655.

\leavevmode\hypertarget{ref-rosen_who_2016}{}%
Rosen, C. C., Koopman, J., Gabriel, A. S., \& Johnson, R. E. (2016). Who
strikes back? A daily investigation of when and why incivility begets
incivility. \emph{Journal of Applied Psychology}, \emph{101}(11), 1620.

\leavevmode\hypertarget{ref-vancouver_translating_2018}{}%
Vancouver, J. B., Wang, M., \& Li, X. (2018). Translating Informal
Theories Into Formal Theories: The Case of the Dynamic Computational
Model of the Integrated Model of Work Motivation. \emph{Organizational
Research Methods}, 109442811878030.
doi:\href{https://doi.org/10.1177/1094428118780308}{10.1177/1094428118780308}

\leavevmode\hypertarget{ref-Wang2016}{}%
Wang, M., Zhou, L., \& Zhang, Z. (2016). Dynamic modeling. \emph{Annual
Review of Organizational Psychology and Organizational Behavior},
\emph{3}(1), 241--266.
doi:\href{https://doi.org/10.1146/annurev-orgpsych-041015-062553}{10.1146/annurev-orgpsych-041015-062553}


\end{document}
